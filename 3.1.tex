\documentclass[12pt, letter]{article}
\usepackage[utf8]{inputenc}
\usepackage[a4paper, total={6in, 8in}]{geometry}
\usepackage{tikz}
\usepackage[T1]{fontenc}
\usepackage{listings}
\usepackage{graphicx}
\usepackage{amsfonts}
\usepackage{amsmath}
\usepackage{amssymb}
\usepackage{amsthm}
\usepackage{mathtools}
\usepackage{listings}
\usepackage{bm}
\newcommand{\uvec}[1]{\boldsymbol{\hat{\textbf{#1}}}}
\usepackage[english]{babel}
\newtheorem{theorem}{Theorem}
\usepackage{setspace}

\setstretch{1.25}
\begin{document}
\section*{Chapter 3}
\section*{Set Theory}
\section*{3.1 Fundamentals}
\subsubsection*{Definition 3.1.1}
(Informal) We define a $set$ $A$ to be any unordered collection of objects, e.g., ${3,8,5,2}$ is a set. If $x$ is an object, we say that $x$ is an element of $A$ or $x\in A$ if $x$ lies in the collection;
otherwise we say that $x\notin A$. For instance, $3\in\{1,2,3,4,5\}$ but $7\notin \{1,2,3,4,5\}$.
\subsubsection*{Axiom 3.1 (Sets are objects).} 
If $A$ is a set, then $A$ is also an object. In particular, given two sets $A$ and $B$, it is meaningful to ask whether $A$ is also an element of $B$.
\subsubsection*{Axiom 3.2 (Equality of sets).}
Two sets $A$ and $B$ are equal, $A=B$, iff every element of $A$ is an element of $B$ and vice versa. To put it another way, $A=B$ if and only if every element $x$ of $A$ belongs also to $B$, and every element $y$
of $B$ belongs also to $A$.
\subsubsection*{Axiom 3.3 (Empty set).}
There exists a set $\emptyset$, known as the empty set, which contains no elements, i.e., for every object $x$ we have $x\notin\emptyset$. 
\subsubsection*{Lemma 3.1.5 (Single choice).}
Let $A$ be a non-empty set. Then there exists an object $x$ such that $x\in A$. 
\subsubsection*{Axiom 3.4 (Singleton sets and pair sets).}
If $a$ is an object, then there exists a set $\{a\}$ whose only element is $a$, i.e., for every object $y$, we have $y\in\{a\}$ if and only if $y=a$; we refer to $\{a\}$ as the singleton set 
whose element is $a$. Furthermore, if $a$ and $b$ are objects, then there exists a set $\{a, b\}$ whose only elements are $a$ and $b$; i.e., for every object $y$, we have $y\in\{a,b\}$ if and only if $y=a$ or $y=b$; 
we refer to this set as the pair set formed by $a$ and $b$.
\subsubsection*{Axiom 3.5 (Pairwise union).}
Given any two sets $A,B$, there exists a set $A\cup B$, called the union of $A$ and $B$, which consists of all the elements which belong to $A$ or $B$ or both. In other words, for any object $x$,
\begin{equation*}
    x\in A\cup B\iff (x\in A \text{ or } x\in B).
\end{equation*}
\subsubsection*{Lemma 3.1.12}
If $a$ and $b$ are objects, then $\{a,b\}=\{a\}\cup\{b\}$. If $A,B,C$ are sets, then the union operation is commutative (i.e., $A\cup B=B\cup A$) and associative (i.e., $(A\cup B)\cup C=A\cup (B\cup C)$). Also,
we have $A\cup A=A\cup\emptyset=\emptyset\cup A=A$.
\subsubsection*{Definition 3.1.14 (Subsets).}
Let $A,B$ be sets. We say that $A$ is a subset of $B$, denoted $A\subseteq B$, iff every element of $A$ is also an element of $B$, i.e.
\begin{equation*}
    \text{For any object }x,\; x\in A\iff x\in B,
\end{equation*} 
We say that $A$ is a proper subset of $B$, denoted $A\subsetneq B$, if $A\subseteq B$ and $A\ne B$.
\subsubsection*{Proposition 3.1.17 (Sets are partially ordered by set inclusion).}
Let $A,B,C$ be sets. If $A\subseteq B$ and $B\subseteq C$ then $A\subseteq C$. If $A\subseteq B$ and $B\subseteq A$, then $A=B$. Finally, if $A\subsetneq B$ and $B\subsetneq C$ then $A\subsetneq C$.
\subsubsection*{Axiom 3.6 (Axiom of specification).}
Let $A$ be a set, and for each $x\in A$, let $P(x)$ be a property pertaining to $x$ (i.e., $P(x)$ is either a true statement or a false statement). Then there exists a set, called 
$\{x\in A: P(x)\text{ is true}\}$ (or simply $\{x\in A: P(x)\}$ for short), whose elements are precisely the elements $x$ in $A$ for which $P(x)$ is true. In other words, for any object $y$,
\begin{equation*}
    y\in\{x\in A: P(x)\text{ is true}\} \iff (y\in A\text{ and }P(y)\text{ is true}).
\end{equation*}
\subsubsection*{Definition 3.1.22 (Intersections).}
The intersection $S_1\cap S_2$ of two sets is defined to be the set 
\begin{equation*}
    S_1\cap S_2:=\{x\in S_1:x\in S_2\}.
\end{equation*}
In other words, $S_1\cap S_2$ consists of all elements which belong to both $S_1$ and $S_2$. Thus, for all objects $x$,
\begin{equation*}
    x\in S_1\cap S_2\iff x\in S_1\text{ and }x\in S_2.
\end{equation*}
\subsubsection*{Definition 3.1.26 (Difference sets).} 
Given two sets $A$ and $B$, we define the set $A-B$ or $A\backslash B$ to be the set $A$ with any elements of $B$ removed:
\begin{equation*}
    A\backslash B:=\{x\in A: x\notin B\};
\end{equation*} 
for instance, $\{1,2,3,4\}\backslash\{2,4,6\}=\{1,3\}$. In many cases $B$ will be a subset of $A$, but not necessarily.
\subsubsection*{Proposition 3.1.27 (Sets form a boolean algebra).}
Let $A,B,C$ be sets, and let $X$ be a set containing $A,B,C$ as subsets.
\begin{enumerate}
    \item (Minimal element) We have $A\cup\emptyset=A$ and $A\cap\emptyset=\emptyset$.
    \item (Maximal element) We have $A\cup X=X$ and $A\cap X=A$.
    \item (Identity) We have $A\cap A=A$ and $A\cup A=A$.
    \item (Commutativity) We have $A\cup B=B\cup A$ and $A\cap B=B\cap A$.
    \item (Associativity) We have $(A\cup B)\cup C=A\cup(B\cup C)$ and $(A\cap B)\cap C=A\cap(B\cap C)$.
    \item (Distributivity) We have $A\cap (B\cup C)=(A\cap B)\cup (A\cap C)$ and $A\cup(B\cap C)=(A\cup B)\cap (A\cup C)$.
    \item (Partition) We have $A\cup(X\backslash A)=X$ and $A\cap (X\backslash A)=\emptyset$.
    \item (De Morgan laws) We have $X\backslash (A\cup B)=(X\backslash A)\cap (X\backslash B)$ and $X\backslash (A\cap B)=(X\backslash A)\cup(X\backslash B)$. 
\end{enumerate}
\subsubsection*{Axiom 3.8 (Infinity).}
There exists a set $\mathbb{N}$, whose elements are clled natural numbers, as well as an object 0 in $\mathbb{N}$, and an object $n++$ assigned to every natural number $n\in\mathbb{N}$, 
such that the Peano axioms hold.



\subsection*{Exercises}
\subsubsection*{Exercise 3.1.1}
Let $a,b,c,d$ be objects such that $\{a,b\}=\{c,d\}$. Show that at least one of the two statements "$a=c$ and $b=d$" and "$a=d$ and $b=c$" hold.
\begin{proof}
    Consider two cases: $a=b$ and $a\ne b$.\\
    Case 1: $a=b$. Then $\{a,b\}=\{a\}$. By Axiom 3.2, if $\{a\}$ and $\{c,d\}$ are equal to each other, then every element belong to $\{c,d\}$ must also belong to $\{a\}$. 
    Therefore, $c=a$, $d=a$. Since $a=b$, we have $a=b=c=d$. Thus, both statements hold.\\
    Case 2: $a\ne b$. Similarly, by Axiom 3.2, every element belong to $\{a,b\}$ must also belong to $\{c,d\}$. So $\{c,d\}$, a set of two elements, contains two distinct elements $a$ and $b$. Therefore, 
    either $a=c, b=d$ or $a=d, b=c$ holds, exclusively.\\
    Thus, we have shown that at least one of the two statements "$a=c$ and $b=d$" and "$a=d$ and $b=c$" hold.
\end{proof}
\subsubsection*{Exercise 3.1.2}
Using only Axiom 3.2, Axiom 3.1, Axiom 3.3, and Axiom 3.4, prove that the sets $\emptyset$, $\{\emptyset\}$, $\{\{\emptyset\}\}$, and $\{\emptyset,\{\emptyset\}\}$ are all distinct.
\begin{proof}
    First, let's consider $\emptyset$. $\emptyset$ contains no element while other sets all have at least one element in it. Therefore, $\emptyset$ is distinct from $\{\emptyset\}$, $\{\{\emptyset\}\}$ and $\{\emptyset,\{\emptyset\}\}$. 
    Then, let's consider $\{\emptyset\}$. Is it distinct from $\{\{\emptyset\}\}$ and $\{\emptyset,\{\emptyset\}\}$?
    We know that $\emptyset\in\{\emptyset\}$. But we have proved earlier $\emptyset$ and $\{\emptyset\}$ are not equal to each other, so $\emptyset\notin \{\{\emptyset\}\}$. So $\{\emptyset\}$ and $\{\{\emptyset\}\}$ are distinct. For the same reason, 
    $\{\emptyset\}\notin \{\emptyset\}$. So $\{\emptyset\}$ and $\{\emptyset,\{\emptyset\}\}$ are also distinct. Last, consider $\{\{\emptyset\}\}$ and $\{\emptyset,\{\emptyset\}\}$. For the same reason ($\emptyset$ and $\{\emptyset\}$ are distinct), 
    $\emptyset\notin \{\{\emptyset\}\}$. So $\{\{\emptyset\}\}$ and $\{\emptyset,\{\emptyset\}\}$ are distinct. Thus, we have proved the sets $\emptyset$, $\{\emptyset\}$, $\{\{\emptyset\}\}$, and $\{\emptyset,\{\emptyset\}\}$ are all distinct.
\end{proof}
\subsubsection*{Exercise 3.1.3}
Prove the remaining claims in Lemma 3.1.12.
\begin{proof}
    First, prove the union operation is commutative (i.e., $A\cup B=B\cup A$). By definition, we know that $A\cup B$ consists of all the elements which belong to $A$ or $B$, inclusively. 
    And $B\cup A$ also consists of all the elements belong to $A$ or $B$, inclusively. Therefore, $A\cup B$ and $B\cup A$ are containing exactly the same elements. Thus, $A\cup B=B\cup A$.
    
    The second part is to prove $A\cup A=A\cup\emptyset=\emptyset\cup A=A$. First, let's consider $A\cup A$. By definition, $A\cup A$ consists of all the element $x$ such that $x\in A$ or $x\in A$. So $A\cup A$ and $A$ have exactly the same elements. Therefore $A\cup A=A$.
    Now let's consider $A\cup\emptyset$. $A\cup\emptyset$ consists of all the $x$ such that $x\in A$ or $x\in\emptyset$. Since no element would belong to $\emptyset$. $A\cup\emptyset$ contains exactly the same elements as $A$. Therefore, $A\cup\emptyset=A$. And by commutative law, 
    we have $A\cup\emptyset=\emptyset\cup A=A$.

    Thus, we have proved $A\cup A=A\cup\emptyset=\emptyset\cup A=A$.
\end{proof}
\subsubsection*{Exercise 3.1.4}
Prove the remaining claims in Lemma 3.1.17.
\begin{proof}
    Part I. If $A\subseteq B$ and $B\subseteq A$, then $A=B$. Translate the if statement into propositional logic: $(x\in A \Rightarrow x\in B) \wedge (x\in B\Rightarrow x\in A)$. Therefore, we have $x\in A\iff x\in B$. Thus, $A=B$.

    Part II. If $A\subsetneq B$ and $B\subsetneq C$ then $A\subsetneq C$. Since $A\ne B$ and $B\ne C$, by transitivity, $A\ne C$. And by the first part of this proposition (if $A\subseteq B$ and $B\subseteq C$ then $A\subseteq C$), we would have $A\subseteq C$. Since $A\subseteq C$ and $A\ne C$, $A\subsetneq C$.
\end{proof}
\subsubsection*{Exercise 3.1.5}
Let $A,B$ be sets. Show that the three statements $A\subseteq B$, $A\cup B=B$, $A\cap B=A$ are logically equivalent (any one of them implies the other two).
\begin{proof}
    Rewrite the statements using propositional logic.

    $A\subseteq B$: $x\in A\Rightarrow x\in B$.

    $A\cup B=B$: $x\in A\lor x\in B\iff x\in B$.

    $A\cap B=A$: $x\in A\land x\in B\iff x\in A$.

    Assume $A\subseteq B$ is true. Then both $x\in A$ and $x\in B$ imply $x\in B$. So $x\in A\lor x\in B\Rightarrow x\in B$ stands. And $x\in B\Rightarrow x\in A\lor x\in B$ stands by rules of inference. Therefore, $A\subseteq B\Rightarrow A\cup B=B$. Assume $A\cup B=B$ is true. Then $x\in A\lor x\in B\Rightarrow x\in B$ stands. So $x\in A\Rightarrow x\in B$ is true which means $A\cup B=B\Rightarrow A\subseteq B$. Therefore, 
    $A\subseteq B\iff A\cup B=B$.

    Assume $A\subseteq B$ is true. We know that $x\in A\land x\in B\Rightarrow x\in A$ is true by rules of inference. And if $x\in A$, since $A\subseteq B$ is true, $x\in B$ is also true. So $x\in A\land x\in B$ is true. Therefore, $A\subseteq B\Rightarrow A\cap B=A$.
    Assume $A\cap B=A$ is true. Then if $x\in A$, $x\in A\land x\in B$ must be true. So $x\in B$ is true. $A\subseteq B$ is true. Therefore, $A\cap B\Rightarrow A\subseteq B$. Thus, $A\subseteq B\iff A\cap B=A$.
    
    By transitivity, we have $A\cup B=B\iff A\cap B=A$. Thus, these three statements are logically equivalent.
\end{proof}
\subsubsection*{Exercise 3.1.6}
Prove Proposition 3.1.27.
\begin{enumerate}
    \item $A\cup\emptyset=A$ and $A\cap\emptyset=\emptyset$.
    \begin{proof}
        $A\cup\emptyset=A$ has been proved in 3.1.3. $A\cap\emptyset$ consists of all $x$ such that $x\in A\land x\in\emptyset$. Since $x\in\emptyset$ is always false, $A\cap\emptyset$ has no element in it. Thus, $A\cap\emptyset=\emptyset$.
    \end{proof}
    \item $A\cup X=X$ and $A\cap X=A$.
    \begin{proof}
        We have proved in 3.1.5 that $A\subseteq X$, $A\cup X=X$ and $A\cap X=A$ are logically equivalent.
    \end{proof}
    \item $A\cap A=A$ and $A\cup A=A$.
    \begin{proof}
        $x\in A\iff x\in A$, so $A\subseteq A$. Therefore, by 3.1.5 we have $A\cap A=A$ and $A\cup A=A$.
    \end{proof}
    \item $A\cup B=B\cup A$ and $A\cap B=B\cap A$.
    \begin{proof}
        $A\cup B=B\cup A$ has been proved in 3.1.3. On the other hand, 
        
        $x\in A\land x\in B\iff x\in B\land x\in A$. Therefore, $A\cap B=B\cap A$.
    \end{proof} 
    \item $(A\cup B)\cup C=A\cup(B\cup C)$ and $(A\cap B)\cap C=A\cap(B\cap C)$.
    \begin{proof}
        $(A\cup B)\cup C=A\cup(B\cup C)$ has been proved in Lemma 3.1.12. 
        
        $x\in A\land(x\in B\land x\in C)\iff (x\in A\land x\in B)\land x\in C$. Hence, $(A\cap B)\cap C=A\cap(B\cap C)$.
    \end{proof}
    \item $A\cap(B\cup C)=(A\cap B)\cup(A\cap C)$ and $A\cup(B\cap C)=(A\cup B)\cap(A\cup C)$.
    \begin{proof}
        This can be proved using distribution law in propositional logic. 

        Since $x\in A\land(x\in B\lor x\in C)\iff(x\in A\land x\in B)\lor (x\in A\lor x\in C)$ and 

        $x\in A\lor(x\in B\land x\in C)\iff(x\in A\lor x\in B)\land(x\in A\lor x\in C)$, $A\cap(B\cup C)=(A\cap B)\cup(A\cap C)$ and $A\cup(B\cap C)=(A\cup B)\cap(A\cup C)$
        are also true.
    \end{proof}
    \item $A\cup(X\backslash A)=X$ and $A\cap(X\backslash A)=\emptyset$.
    \begin{proof}
        $A\cup(X\backslash A)\iff (x\in A\lor(x\in X\land x\notin A))\iff(x\in A\lor x\in X)\iff A\cup X$.
        From 3.1.5, we have $A\cup X=X$. Thus, $A\cup(X\backslash A)=X$.

        $A\cap(X\backslash A)\iff x\in A\land(x\in X\land x\notin A)\iff (x\in A\land x\notin A)\land x\in X$. $(x\in A\land x\notin A)\land x\in X$ is always false. Therefore, 
        $A\cap(X\backslash A)=\emptyset$.
    \end{proof}
    \item $X\backslash(A\cup B)=(X\backslash A)\cap(X\backslash B)$ and $X\backslash(A\cap B)=(X\backslash A)\cup(X\backslash B)$.
    \begin{proof}
        $X\backslash(A\cup B)\iff x\in X\land\sim(x\in A\lor x\in B)\iff (x\in X\land x\notin A)\land (x\in X\land x\notin B)\iff (X\backslash A)\cap(X\backslash B)$.

        $X\backslash (A\cap B)\iff x\in X\land\sim(x\in A\land x\in B)\iff x\in X\land(x\notin A\lor x\notin B)\iff(x\in X\land x\notin A)\lor(x\in X\land x\in B)\iff (X\backslash A)\cup(X\backslash B)$.
    \end{proof}
\end{enumerate}
\end{document}