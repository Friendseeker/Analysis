\documentclass[12pt, letter]{article}
\usepackage[utf8]{inputenc}
\usepackage[a4paper, total={6in, 8in}]{geometry}
\usepackage{tikz}
\usepackage[T1]{fontenc}
\usepackage{listings}
\usepackage{graphicx}
\usepackage{amsfonts}
\usepackage{amsmath}
\usepackage{amssymb}
\usepackage{amsthm}
\usepackage{mathtools}
\usepackage{listings}
\usepackage{bm}
\newcommand{\uvec}[1]{\boldsymbol{\hat{\textbf{#1}}}}
\usepackage[english]{babel}
\newtheorem{theorem}{Theorem}
\usepackage{setspace}

\setstretch{1.25}
\begin{document}
\section*{3.5 Cartesian products}
\subsubsection*{Definition 3.5.1 (Ordered pair).}
If $x$ and $y$ are any objects (possibly equal), we define the ordered pair $(x,y)$ to be a new object, consisting of $x$
as its first component and $y$ as its second component. Two ordered pairs $(x,y)$ and $(x',y')$ are considered equal if and only if 
both their components match, i.e.
\begin{equation*}
    (x,y)=(x',y')\iff (x=x'\text{ and }y=y').
\end{equation*}
\subsubsection*{Definition 3.5.4 (Cartesian product).}
If $X$ and $Y$ are sets, then we define the Cartesian product $X\times Y$ to be the collection of ordered pairs, whose first component
lies in $X$ and second component lies if $Y$, thus
\begin{equation*}
    X\times Y=\{(x,y):x\in X, y\in Y\}
\end{equation*}
or equivalently
\begin{equation*}
    a\in (X\times Y)\iff (a=(x,y) \text{ for some } x\in X and y\in Y).
\end{equation*}
\subsubsection*{Definition 3.5.7 (Ordered $n$-tuple and $n$-fold Cartesian product).}
Let $n$ be a natural number. An ordered $n$-tuple $(x_i)_{1\leq i\leq n}$ (also denoted $(x_1,\cdots, x_n)$) is a collection of objects $x_i$,
one for every natural number $i$ between $1$ and $n$; we refer to $x_i$ as the $i^{th}$ component of the ordered $n$-tuple. Two 
ordered $n$-tuples $(x_i)_{1\leq i\leq n}$ and $(y_i)_{1\leq i\leq n}$ are said to be equal iff $x_i=y_i$ for all $1\leq i\leq n$. 
If $(X_i)_{1\leq i\leq n}$ is an ordered $n$-tuple of sets, we define their Cartesian product $\prod_{1\leq i\leq n}X_i$ (also denoted 
$\prod_{i=1}^{n}X_i$ or $X_1\times \cdots \times X_n$) by
\begin{equation*}
    \prod_{i\leq i\leq n}X_i:=\{(x_i)_{1\leq i\leq n}:x_i\in X_i\text{ for all }1\leq i\leq n\}.
\end{equation*}
\subsubsection*{Lemma 3.5.12 (Finite choice).}
Let $n\geq 1$ be a natural number, and for each natural number $1\leq i\leq n$, let $X_i$ be a non-empty set. Then there exists an $n$-tuple $(x_i)_{1\leq i\leq n}$
such that $x_i\in X_i$ for all $1\leq i\leq n$. In other words, if each $X_i$ is non-empty, then the set $\prod_{1\leq i\leq n}X_i$
is also non-empty.
\subsection*{Exercises}
\subsubsection*{Exercise 3.5.1}
Suppose we define the ordered pair $(x,y)$ for any objects $x$ and $y$ by the formula $(x,y):=\{\{x\},\{x,y\}\}$ (thus using several applications of Axiom 3.4).
Thus for instance $(1,2)$ is the set $\{\{1\},\{1,2\}\}$, $(2,1)$ is the set $\{\{2\},\{2,1\}\}$, and $(1,1)$ is the set $\{1\}$.
Show that such a definition indeed obeys the property $(3.5)$, and also whenever $X$ and $Y$ are sets, the Cartesian product $X\times Y$ is also a set. 
Thus this definition can be validly used as a definition of an ordered pair. For an additional challenge, show that the alternate definition $(x,y):=\{x,\{x,y\}\}
$ also verifies $(3.5)$ and is thus also an acceptable definition of ordered pair.
\begin{enumerate}
    \item Show that $(x,y):=\{\{x\},\{x,y\}\}$ is a valid definition of an ordered pair.
    \begin{proof}
        First, we need to show that 
        \begin{equation*}
            (x,y)=(x',y')\iff (x=x'\text{ and }y=y').
        \end{equation*} 

        Suppose $(x,y)=(x',y')$. Then by definition, $\{\{x\},\{x,y\}\}=\{\{x'\},\{x',y'\}\}$. If $x=y$, $\{\{x\},\{x,y\}\}=\{\{x\},\{x\}\}=\{\{x\}\}$. 
        Then $\{\{x'\},\{x',y'\}\}$ must also only contain one element. Thus, $\{x'\}=\{x',y'\}$. So $x'=y'$. Lastly, we have $\{\{x\}\}=\{\{x'\}\}$. So $x=x'$. 
        Thus, $x=x'=y=y'$. It is the same thing if we assume $x'=y'$ at first. Now consider the case $x\ne y$. Then 
        $\{x\}$ has one element and $\{x,y\}$ has two elements. And since $\{x'\}$ could only contain one element, we have the following relations:
        \begin{equation*}
            \begin{cases}
                \{x\}=\{x'\}\implies x=x'\\
                \{x,y\}=\{x',y'\}\text{ and }x=x' \implies y=y'.
            \end{cases}
        \end{equation*}
        It would be the same if we assume $x'\ne y'$. Thus, $(x,y)=(x',y')\implies (x=x'\text{ and }y=y')$

        Suppose $x=x'$ and $y=y'$. Then we must have $\{x\}=\{x'\}$ and $\{x,y\}=\{x',y'\}$. Hence, $\{\{x\},\{x,y\}\}=\{\{x'\},\{x',y'\}\}$. 
        Thus, $(x=x'\text{ and }y=y')\implies (x,y)=(x',y')$.

        Therefore, $(x,y)=(x',y')\iff (x=x'\text{ and }y=y')$. This definition verifies $(3.5)$. 
    
        Then, we need to show that whenever $X$ and $Y$ are sets, the Cartesian product $X\times Y$ is a set.
        Use the definition above: $(x,y)=\{\{x\},\{x,y\}\}$. The powerset of $X\cup Y$ is $\{0,\{x\},\{y\},\{x,y\}\}$ which contains the elements in 
        $(x,y)$. Then the powerset of the powerset of $X\cup Y$ contains $(x,y)$. The elements in $\mathcal{P}(\mathcal{P}(X\cup Y))$ is a set, thus, the Cartesian 
        product is a set. More specifically, 
        \begin{equation*}
            X\times Y=\{z\in \mathcal{P}(\mathcal{P}(X\cup Y)):z \text{ contains exactly one singleton set and one pair set}\}.
        \end{equation*}
    \end{proof}
    \item Show that $(x,y)=\{x,\{x,y\}\}$ is also a valid definition of an ordered pair.
    \begin{proof}
        We need to show that 
    \begin{equation*}
        \{x,\{x,y\}\}=\{x',\{x',y'\}\}\iff x=x'\text{ and }y=y'.    
    \end{equation*}
    Suppose $\{x,\{x,y\}\}=\{x',\{x',y'\}\}$. Denote $A=\{x,y\}$, $B=\{x',y'\}$. Then $\{x,A\}=\{x',B\}$. Since sets are objects, $\{x,A\}$ and 
   $\{x',B\}$ are both pair sets. Since $x\in \{x,\{x,y\}\}$ and $x\in\{x,\{x,y\}\}\implies x\in\{x',\{x',y'\}\}$. So either $x=x'$ or $x=\{x',y'\}$.
   Assume $x=\{x',y'\}$. Then the only option left is $x'=\{x,y\}$. As $x$ and $x'$ are both sets, having $x\in x'$ and $x'\in x$ at the same time violates
   the statements in Exercise 3.2.2. Therefore, $x=x'$ and $\{x,y\}=\{x',y'\}=\{x,y'\}$. Thus, $y=y'$. 
   
   Suppose $x=x'$ and $y=y'$. Then clearly we have $\{x,y\}=\{x',y'\}$. So $\{x,\{x,y\}\}=\{x',\{x',y'\}\}$. 

   Thus, $(x,y):=\{x,\{x,y\}\}$ verifies $(3.5)$.
    \end{proof}
\end{enumerate}
\subsubsection*{Exercise 3.5.2}
Suppose we define an ordered $n$-tuple to be a surjective function $x:\{i\in \mathbb{N}:1\leq i\leq n= n\}\to X$ whose range is some arbitrary set $X$
(so different ordered $n$-tuples are allowed to have different ranges); we then write $x_i$ for $x(i)$, and also write $x$ as $(x_i)_{1\leq i\leq n}$.
Using this definition, verify that we have $(x_i)_{1\leq i\leq n}=(y_i)_{1\leq i\leq n}$ if and only if $x_i=y_i$ for all $1\leq i\leq n$. Also,
show that if $(X_i)_{1\leq i\leq n}$ is an ordered $n$-tuple of sets, then the Cartesian product, as defined in Definition 3.5.7, is indeed a set.
\begin{enumerate}
    \item $(x_i)_{1\leq i\leq n}=(y_i)_{1\leq i\leq n}\iff x_i=y_i$ for all $1\leq i\leq n$.
    \begin{proof}
        Apparently, $x$ and $y$ have the same domain $\{i\in\mathbb{N}:1\leq i\leq n\}$. Suppose $y:\{i\in \mathbb{N}:1\leq i\leq n\}\to Y$. 

        Suppose $(x_i)_{1\leq i\leq n}=(y_i)_{1\leq i\leq n}$. Since $x$ and $y$ are two functions, we must have $X$=$Y$ so that they have the same range.
        And by Definition 3.3.7, we have $x(i)=y(i)$ for all $1\leq i\leq n$. Therefore, $(x_i)_{1\leq i\leq n}=(y_i)_{1\leq i\leq n}\implies x_i=y_i$ for all $1\leq i\leq n$..
    
        Suppose $x_i=y_i$ for all $1\leq i\leq n$. Since $x$ and $y$ are both surjective and $\{x_1,\cdots, x_n\}=\{y_1,\cdots, y_n\}$, $X=Y=\{x_1,\cdots, x_n\}=\{y_1,\cdots, y_n\}$. 
        Thus, $x$ and $y$ have the same range. And because they also have the same domain and $x_i=y_i$ for all $1\leq i\leq n$, $x=y$. Therefore, we have proved $x_i=y_i$ for all $1\leq i\leq n
        \implies x=y$. 
        
        Thus, $(x_i)_{1\leq i\leq n}=(y_i)_{1\leq i\leq n}\iff x_i=y_i$ for all $1\leq i\leq n$.
    \end{proof} 
    \item If $(X_i)_{1\leq i\leq n}$ is an ordered $n$-tuple of sets, then the Cartesian product is a set.
    \begin{proof}
        Denote $A=\{X_1, X_2,\cdots, X_n\}$, so every element of $A$ is a set itself and by the union axiom we have $\bigcup A$ being the set consists of all the elements of the elements of $A$. 
        Denote $I=\{i\in \mathbb{N}: 1\leq i\leq n\}$. Then, the mapping function $x$ would be partial functions with domain $I$ which is also a subset of $I$ and range being a subset of $\bigcup A$. Denote it as $X$. 
        Thus, by Exercise 3.4.7, the collection of all these partial functions is a set. By Definition 3.5.7, the Cartesian product would be a subset of the set of all these partial functions.
        Let $P(x)$ be $x_i\in X_i$ for all $1\leq i\leq n$. By Axiom of specification, there exists a set $\{x\in X: P(x)\text{ is true}\}$ which is the same 
        as the Cartesian product. Therefore, the Cartesian product is indeed a set.
    \end{proof}
\end{enumerate}
\subsubsection*{Exercise 3.5.3}
Show that the definitions of equality for ordered pair and ordered $n$-tuple obey the reflexivity, symmetry, and transitivity axioms.
\begin{itemize}
    \item reflexivity
    \begin{proof}
        For the ordered pair $(x,y)$, since $x=x$ and $y=y$, we have $(x,y)=(x,y)$. 
        For the ordered $n$-tuple $(x_i)_{1\leq i\leq n}$, since $x_i=x_i$ for $1\leq i\leq n$, by definition, we have $(x_i)_{1\leq i\leq n}$.     
    \end{proof}  
    \item symmetry
    \begin{proof}
        We want to show $(x,y)=(x',y')\iff (x',y')=(x,y)$. Assume $(x,y)=(x',y')$. Then $x=x'$ and $y=y'$. By symmetry property of equality, we have 
        $x'=x$ and $y'=y$. By definition, we have $(x',y')=(x,y)$. Similarly, we can show that $(x',y')=(x,y)\implies (x,y)=(x',y')$.
        Thus, $(x,y)=(x',y')\iff (x',y')=(x,y)$.

        For ordered $n$-tuple, we want to show that $(x_i)_{1\leq i\leq n}=(y_i)_{1\leq i\leq n}\iff(y_i)_{1\leq i\leq n}= (x_i)_{1\leq i\leq n}$.
        Assume $(x_i)_{1\leq i\leq n}=(y_i)_{1\leq i\leq n}$. Then $x_i=y_i$ for every $1\leq i\leq n$. By the symmetry property of equality, we have 
        $y_i=x_i$ for every $1\leq i\leq n$. Therefore, by definition of ordered $n$-tuple, we have $(y_i)_{1\leq i\leq n}= (x_i)_{1\leq i\leq n}$. 
        The approach is the same for the other way around. Thus, $(x_i)_{1\leq i\leq n}=(y_i)_{1\leq i\leq n}\iff(y_i)_{1\leq i\leq n}= (x_i)_{1\leq i\leq n}$.
    \end{proof}  
    \item transitivity 
    \begin{proof}
        The proof for ordered pair is omitted since it is only a special case of ordered $n-$tuple. 
        We need to show that $(x_i)_{1\leq i\leq n}=(y_i)_{1\leq i\leq n}\text{ and } (y_i)_{1\leq i\leq n}=(z_i)_{1\leq i\leq n} \implies 
        (x_i)_{1\leq i\leq n}=(z_i)_{1\leq i\leq n}$. Since $(x_i)_{1\leq i\leq n}=(y_i)_{1\leq i\leq n}$, we have $x_i=y_i$ for $1\leq i\leq n$. 
        Since $(y_i)_{1\leq i\leq n}=(z_i)_{1\leq i\leq n}$, we have $y_i=z_i$ for $1\leq i\leq n$. By transitivity property of equality, we have $x_i=z_i$. 
        By definition of ordered $n$-tuple, $(x_i)_{1\leq i\leq n}=(z_i)_{1\leq i\leq n}$. 
    \end{proof}
\end{itemize}
\subsubsection*{Exercise 3.5.4}
Let $A$, $B$, $C$ be sets. Show that $A\times (B\cup C)=(A\times B)\cup (A\times C)$, that $A\times (B\cap C)=(A\times B)\cap (A\times C)$, and 
that $A\times (B\backslash C)=(A\times B)\backslash(A\times C)$.
\begin{enumerate}
    \item $A\times (B\cup C)=(A\times B)\cup (A\times C)$.
    \begin{proof}
        We need to show that $(x,y)\in A\times (B\cup C)\iff (x,y)\in (A\times B)\cup (A\times C)$.

        Suppose $(x,y)\in A\times (B\cup C)$. By definition, we have $x\in A$ and $y\in (B\cup C)$. $y\in B\cup C\iff y\in B\text{ or }y\in C$. 
        Therefore, $(x,y)\in (A\times B)$ or $(x,y)\in (A\times C)$. Hence, $(x,y)\in (A\times B)\cup (A\times C)$. 

        Suppose $(x,y)\in (A\times B)\cup (A\times C)$. Then we have either $(x,y)\in A\times B$ or $(x,y)\in A\times C$. 
        $(x,y)\in A\times B\implies x\in A\text{ and }y\in B$. $(x,y)\in A\times C\implies x\in A\text{ and }y\in C$. Therefore, we have $x\in A$ and $y\in B\cup C$. 
        Hence, $(x,y)\in A\times (B\cup C)$. 

        Thus, $A\times (B\cup C)=(A\times B)\cup (A\times C)$.
    \end{proof}
    \item $A\times (B\cap C)=(A\times B)\cap (A\times C)$.
    \begin{proof}
        We need to show that $(x,y)\in A\times (B\cap C)\iff (x,y)\in (A\times B)\cap (A\times C)$.

        Suppose $(x,y)\in A\times (B\cap C)$. Then $x\in A$ and $y\in B\cap C$. $y\in B\cap C\iff y\in B\text{ and }y\in C$. Then we have $(x,y)\in A\times B$ and 
        $(x,y)\in A\times C$. Hence, $(x,y)\in (A\times B)\cap (A\times C)$. 

        Suppose $(x,y)\in (A\times B)\cap (A\times C)$. Then $(x,y)\in A\times B$ and $(x,y)\in A\times C$. $(x,y)\in A\times B\implies x\in A$ and $y\in B$. 
        $(x,y)\in A\times C\implies x\in A$ and $y\in C$. Overall, we have $x\in A$ and $y\in B\cap C$. Hence, $(x,y)\in A\times (B\cap C)$.
        
        Thus, $A\times (B\cap C)=(A\times B)\cap (A\times C)$.
    \end{proof}
    \item $A\times (B\backslash C)=(A\times B)\backslash (A\times C)$.
    \begin{proof}
        We need to show that $(x,y)\in A\times (B\backslash C)\iff (x,y)\in (A\times B)\backslash (A\times C)$.

        Suppose $(x,y)\in A\times (B\backslash C)$. Then $x\in A$ and $y\in B$ and $y\notin C$. $x\in A$ and $y\in B\implies (x,y)\in A\times B$.
        $y\notin C\implies (x,y)\notin A\times C$. Therefore, $(x,y)\in (A\times B)\backslash (A\times C)$.

        Suppose $(x,y)\in (A\times B)\backslash (A\times C)$. Then $(x,y)\in A\times B$ and $(x,y)\notin A\times C$. $(x,y)\in A\times B\implies x\in A$ and $y\in B$. 
        $(x,y)\notin A\times C$ and $x\in A\implies y\notin C$. Therefore, we have $x\in A$ and $y\in B$ and $y\notin C$. Hence, $(x,y)\in A\times (B\backslash C)$.

        Thus, $A\times (B\backslash C)=(A\times B)\backslash (A\times C)$.
    \end{proof}
\end{enumerate}
\subsubsection*{Exercise 3.5.5}
Let $A$, $B$, $C$, $D$ be sets. Show that $(A\times B)\cap (C\times D)=(A\cap C)\times (B\cap D)$. Is it true that $(A\times B)\cup (C\times D)=(A\cup C)\times (B\cup D)$?
Is it true that $(A\times B)\backslash (C\times D)=(A\backslash C)\times (B\backslash D)$?
\begin{proof}
    $(A\times B)\cap (C\times D)=(A\cap C)\times (B\cap D)\iff ((x,y)\in(A\times B)\cap (C\times D)\iff (x,y)\in (A\cap C)\times (B\cap D))$.
    Suppose $x\in (A\times B)\cap (C\times D)$. $(x,y)\in A\times B\implies x\in A$ and $y\in B$. $(x,y)\in C\times D\implies x\in C$ and $y\in D$. Therefore, we have
    $x\in A\cap C$ and $y\in B\times D$. Thus, $(x,y)\in (A\cap C)\times (B\cap D)$.
    Suppose $(x,y)\in (A\cap C)\times (B\cap D)$. Then $x\in A\cap C$ and $y\in B\cap D$. $x\in A\cap C\implies x\in A$ and $x\in C$. $y\in B\cap D\implies y\in B$ and $y\in D$.
    $x\in A$ and $y\in B\implies (x,y)\in A\times B$. $x\in C$ and $y\in D\implies (x,y)\in C\times D$. Hence, $(x,y)\in (A\times B)\cap (C\times D)$. 
    Thus, $(A\times B)\cap (C\times D)=(A\cap C)\times (B\cap D)$. 

    $(A\times B)\cup (C\times D)=(A\cup C)\times (B\cup D)$ is not true. Assume $(x,y)\in (A\cup C)\times (B\cup D)$. And suppose $x\in A$ and $y\in D$.
    Then $(x,y)\in A\times D$ and $(x,y)\notin A\times B$ and $(x,y)\notin C\times D$. Hence, $x\notin (A\times B)\cup (C\times D)$. Thus, $(A\times B)\cup (C\times D)=(A\cup C)\times (B\cup D)$ is not true.

    $(A\times B)\backslash (C\times D)=(A\backslash C)\times (B\backslash D)$ is not true. A counterexample: $x\in A\cap C$ and $y\in B\backslash D$. Then 
    $(x,y)\in(A\times B)\backslash (C\times D)$ but $(x,y)\notin (A\backslash C)\times (B\backslash D)$. Thus, $(A\times B)\backslash (C\times D)=(A\backslash C)\times (B\backslash D)$ is not true.
\end{proof}
\subsubsection*{Exercise 3.5.6}
Let $A,B,C,D$ be non-empty sets. Show that $A\times B\subseteq C\times D$ if and only if $A\subseteq C$ and $B\subseteq D$, and that $A\times B=C\times D$ if and only if $A=C$ and $B=D$. 
What happens if the hypotheses that the $A,B,C,D$ are all non-empty are removed?
\begin{enumerate}
    \item $A\times B\subseteq C\times D\iff A\subseteq C$ and $B\subseteq D$.
    \begin{proof}
        Suppose $A\times B\subseteq C\times D$, that is $(x,y)\in A\times B\implies (x,y)\in C\times D$. Since $A,B,C,D$ are non-empty, we have two conditions: $x\in A\implies x\in C$ and
        $y\in B\implies y\in D$. Thus, $A\subseteq C$ and $B\subseteq D$.

        Suppose $A\subseteq C$ and $B\subseteq D$. Then we have $x\in A\implies x\in C$ and $y\in B\implies y\in D$. Combining these two conditions, $(x,y)\in A\times B\implies (x,y)\in C\times D$.
        Hnece, $A\times B\subseteq C\times D$.

        Thus, $A\times B\subseteq C\times D\iff A\subseteq C$ and $B\subseteq D$.
    \end{proof}
    \item $A\times B=C\times D\iff A=C$ and $B=D$.
    \begin{proof}
        Suppose $A\times B=C\times D$. Then $(x,y)\in A\times B\iff (x,y)\in C\times D$. Since $A,B,C,D$ are non-empty, we have $x\in A\iff x\in C$ and 
        $y\in B\iff y\in D$. Therefore, $A=C$ and $B=D$.

        Suppose $A=C$ and $B=D$. Then $x\in A\iff x\in C$ and $y\in B\iff y\in D$. Therefore, $(x,y)\in A\times B\iff (x,y)\in C\times D$. Hence, $A\times B=C\times D$.

        Thus, $A\times B=C\times D\iff A=C$ and $B=D$.
    \end{proof}
\end{enumerate}
If the hypotheses that the $A,B,C,D$ are all non-empty are removed, the equalities will not hold any more. A counterexample would be $A$ is non-empty, $B=\emptyset$, $C=\emptyset$, and $D$ is non-empty. 
$A\times B\subseteq C\times D$ but $A$ is not a subset of $C$. $A\times B=C\times D$ but $A\ne C$.
\subsubsection*{Exercise 3.5.7}
Let $X,Y$ be sets, and let $\pi_{X\times Y\to X}:X\times Y\to X$ and $\pi_{X\times Y\to Y}:X\times Y\to Y$ be the maps $\pi_{X\times Y\to X}(x,y):=x$ and
$\pi_{X\times Y\to Y}(x,y):=y$; these maps are known as the co-ordinate functions on $X\times Y$. Show that for any functions $f:Z\times X$ and $g:Z\times Y$, there exists a unique
function $h:Z\to X\times Y$ such $\pi_{X\times Y\to X}\circ h=f$ and $\pi_{X\times Y\to Y}\circ h=g$. This fuctions $h$ is known as the direct sum of $f$ and $g$ and is denoted $h=f\oplus g$.
\begin{proof}
    Fisrt, prove the existence of $h$. Let $h$ be $h(z):=(f(z),g(z))$. Then $f$ and $\pi_{X\times Y\to X}\circ h$ both have domain $Z$ and range $X$. And $g$ and $\pi_{X\times Y\to Y}\circ h$
    both have domain $Z$ and range $Y$. Consider an arbitrary $z\in Z$. $(\pi_{X\times Y\to X}\circ h)(z)=\pi_{X\times Y\to X}(h(z))=\pi_{X\times Y\to X}(f(z),g(z))=f(z)$,
    $(\pi_{X\times Y\to Y}\circ h)(z)=\pi_{X\times Y\to Y}(h(z))=\pi_{X\times Y\to Y}(f(z),g(z))=g(z)$. Thus, there exists a function $h:Z\to X\times Y$ such that 
    $\pi_{X\times Y\to X}\circ h=f$ and $\pi_{X\times Y\to Y}\circ h=g$. Then, prove the uniqueness of $h$. Suppose there exists $h':Z\to X\times Y$ such that $\pi_{X\times Y\to X}\circ h'=f$ and $\pi_{X\times Y\to Y}\circ h'=g$. 
    Assume $h'(z)=(f'(z),g'(z))$ where $f':Z\to X$ and $g':Z\to Y$. Then $h$ and $h'$ both have domain $Z$ and range $X\times Y$. Consider an arbitrary $z\in Z$, we have $\pi_{X\times Y\to X}(h'(z))=f'(z)=f(z)$
    and $\pi_{X\times Y\to Y}(h'(z))=g'(z)=g(z)$. Therefore, $f'=f$ and $g'=g$. So $h'=h$. Hence, there exists a unique function $h:Z\to X\times Y$ such $\pi_{X\times Y\to X}\circ h=f$ and $\pi_{X\times Y\to Y}\circ h=g$.
\end{proof}
\subsubsection*{Exercise 3.5.8}
Let $X_1,\dotsc,X_n$ be sets. Show that the Cartesian product $\prod_{i=1}^n X_i$ is empty if and only if at least one of the $X_i$ is empty.
\begin{proof}
    We need to show that $\prod_{i=1}^n X_i$ is empty $\iff$ at least one of the $X_i$ is empty.

    Suppose $\prod_{i=1}^n X_i$ is empty and assume each of $X_i$ is non-empty. Then we can find an object $x_i\in X_i$ for all $1\leq i\leq n$. Therefore, there exists $(x_i)_{1\leq i\leq n}\in \prod_{i=1}^n X_i$. Thus,
    $\prod_{i=1}^n$ is non-empty. Contradiction. Hence, at least one of the $X_i$ is empty.

    Suppose at least one of the $X_i$ is empty. Assume $X_i$ is empty for some $1\leq i\leq n$. Then there does not exist an object $x_i$ such that $x_i\in X_i$. By definition of the Cartesian product,
    being an obejct of $\prod_{i=1}^n$ requires $x_i\in X_i$ for all $1\leq i\leq n$. Thus, such a set of $X_i$ can not fulfill the requirement. Hence, $\prod_{i=1}^n$ is empty.
    
    Thus, $\prod_{i=1}^n X_i$ is empty $\iff$ at least one of the $X_i$ is empty.
\end{proof} 
\subsubsection*{Exercise 3.5.9}
Suppose that $I$ and $J$ are two sets, and for all $\alpha\in I$ let $A_{\alpha}$ be a set, and for all $\beta\in J$ let $B_\beta$ be a set. Show that $(\bigcup_{\alpha\in I}A_{\alpha})\cap (\bigcup_{\beta\in J}B_\beta)=\bigcup_{(\alpha,\beta)\in I\times J}(A_{\alpha}\cap B_\beta)$.
\begin{proof}
    We need to show that $x\in (\bigcup_{\alpha\in I}A_{\alpha})\cap (\bigcup_{\beta\in J}B_\beta)\iff x\in \bigcup_{(\alpha,\beta)\in I\times J}(A_{\alpha}\cap B_\beta)$.

    Suppose $x\in (\bigcup_{\alpha\in I}A_{\alpha})\cap (\bigcup_{\beta\in J}B_\beta)$. Then for some $\alpha\in I$, $x\in A_\alpha$, and for some $\beta\in J$, $x\in B_\beta$. By the definition of ordered pair,
    we have for some $(\alpha, \beta)\in I\times J$, $x\in A_\alpha\cap B_\beta$. Hence, $x\in \bigcup_{(\alpha,\beta)\in I\times J}(A_{\alpha}\cap B_\beta)$.

    Suppose $x\in\bigcup_{(\alpha,\beta)\in I\times J}(A_{\alpha}\cap B_\beta)$. Then for some $(\alpha,\beta)\in I\times J$, $x\in A_\alpha\cap B_\beta$. By the definition of ordered pair, there exists some $\alpha\in I$ such that $x\in A_\alpha$ and 
    some $\beta\in J$ such that $x\in B_\beta$. Therefore, $x\in (\bigcup_{\alpha\in I}A_{\alpha})\cap (\bigcup_{\beta\in J}B_\beta)$.

    Thus, $x\in (\bigcup_{\alpha\in I}A_{\alpha})\cap (\bigcup_{\beta\in J}B_\beta)\iff x\in \bigcup_{(\alpha,\beta)\in I\times J}(A_{\alpha}\cap B_\beta)$.
\end{proof}
\subsubsection*{Exercise 3.5.10}
If $f:X\to Y$ is a function, define the graph of $f$ to be the subset of $X\times Y$ defined by $\{(x,f(x)):x\in X\}$. Show that two functions $f:X\to Y$ and $\tilde{f}:X\to Y$ are equal if and only if they have the same graph. Conversely, if $G$ is any subset of $X\times Y$
with the property that for each $x\in X$, the set $\{y\in Y:(x,y)\in G\}$ has exactly one element (or in other words, $G$ obeys the vertical line test), show that there is exactly one function $f:X\to Y$ whose graph is equal to $G$.
\begin{proof}
    We want to show that $f=\tilde{f}\iff\{(x,f(x)):x\in X\}=\{(x,\tilde{f}(x)):x\in X\}$.

    Suppose $f=\tilde{f}$. Then for every $x\in X$, we have $f(x)=\tilde{f}(x)$. Consider an arbitrary $x\in X$, since $f(x)=\tilde{f}(x)$, we have $(x,f(x))=(x,\tilde{f}(x))$. So for every $x\in X$, $(x,\tilde{f}(x))\in\{(x,f(x)):x\in X\}$ and 
    $(x,f(x))\in\{(x,\tilde{f}(x)):x\in X\}$. Hence, $\{(x,f(x)):x\in X\}=\{(x,\tilde{f}(x)):x\in X\}$.

    Suppose $\{(x,f(x)):x\in X\}=\{(x,\tilde{f}(x)):x\in X\}$. Then for an arbitrary $x\in X$, $(x,f(x))\in\{(x,\tilde{f}(x)):x\in X\}$. And for this specific $x\in X$, we have $(x,\tilde{f}(x))\in\{(x,\tilde{f}(x)):x\in X\}$. 
    By the definition of function, for every $x\in X$, there exists exactly one $y\in Y$ such that $\tilde{f}(x)=y$. Hence, $f(x)=\tilde{f}(x)$. As $f$ and $\tilde{f}$ have the same domain and range, $f$ = $\tilde{f}$.

    Thus, $f:X\to Y$ and $\tilde{f}:X\to Y$ are equal if and only if they have the same graph.

    Let $P(x,y)$ be $(x,y)\in G$. Then for every $x\in X$, there exists exactly one $y\in Y$ such that $P(x,y)$ is true. The vertical line test for $G$ reflects that there exists a function $f$ such that for every $x\in X, f(x)\in\{y\in Y:(x,y)\in G\}$.
    The next step is to show that the graph of $f$ is equal to $G$. Since for every $x\in X$, the set $\{y\in Y:(x,y)\in G\}$ has exactly one element and this element, by definition, is $f(x)$, we then can replace $y$ with $f(x)$. Then for every $x\in X$, 
    there exist exactly one $y\in Y$ such that $(x,y)=(x,f(x))\in G$. Therefore, $G$ is the collection of $(x,f(x))$ for all $x\in X$ which can be written as $\{(x,f(x)):x\in X\}$. Thus, the graph of $f$ is equal to $G$.
    Then, we need to show the uniqueness of $f$. Assume there is another function $f':X\to Y$ whose graph is $G$. For an arbitrary $x$, by definition, $(x,f'(x))\in G$. And the graph of $f$ is also $G$, $(x,f(x))\in G$. Since for each $x\in X$, the set 
    $\{y\in Y:(x,y)\in G\}$ has only one element, we must have $f(x)=f'(x)$. Hence, $f=f'$. Thus, there is exactly one function $f:X\to Y$ whose graph is equal to $G$.
\end{proof} 
\subsubsection*{Exercise 3.5.11}
Show that Axiom 3.11 can in fact be deduced from Lemma 3.4.10 and the other axioms of set theory, and thus Lemma 3.4.10 can be used as an 
alternate formulation of the power set axiom. 
\begin{proof}
    First, we need to construct a set of all subsets of $X\times Y$ which obey the vertical line test. By lemma 3.4.10, we know that there exists a set $Z$ consists of all the subsets of $X\times Y$. Let $P(z)$ be $z$ obeys the vertical line test. Then 
    by Axiom of specification, there exists a set
    \begin{equation*}
        G=\{z\in Z: P(z) \text{ is true}\}.
    \end{equation*}
    So this set is the set of all subsets of $X\times Y$ which obey the vertical line test, denote it as $G$. Let $P(f,g)$ be $f$ is exactly the function $f:X\to Y$ whose graph is equal to $g$. By Axiom of replacement, there exists a set 
    \begin{equation*}
        F=\{f:P(f,g)\text{ is true for some }g\in G\}.
    \end{equation*}
    This set should be the set consists of all the functions from $X$ to $Y$. The last step is to justify that this set indeed contains all the functions from $X$ to $Y$. The graph of any function $f: X\to Y$ is a subset of $X\times Y$: 
    as stated in 3.5.10, if $f:X\to Y$, the graph of $f$ is defined to be the subset of $X\to Y$ that is $\{(x,f(x)):x\in X\}$. So set $G$ contains the graphs of all $f:X\to Y$. Then we need to show that the inverse map does not lose any information, in other words,
    $F$ contains all the functions $f:X\to Y$. Since for every $g\in G$ there exists exactly one function whose graph is equal to $g$ and $f:X\to Y$, $\tilde{f}:X\to Y$ are equal if and only if they have the same graph, it is not possible to have two different graphs mapped 
    to a same function $f$. Thus, $G$ contains all the graphs is equivalent to $f$ contains all the functions from $X$ to $Y$. Hence, Axiom 3.11 is deduced from Lemma 3.4.10 and other axioms of set theory. 
\end{proof}
\subsubsection*{Exercise 3.5.12}
This exercise will establish a rigorous version of Proposition 2.1.16. Let $f:\mathbb{N}\times \mathbb{N}\to \mathbb{N}$ be a function, and let $c$ be a natural number. Show that there exists a function $a:\mathbb{N}\to \mathbb{N}$ such that 
\begin{equation*}
    a(0)=c
\end{equation*}
and 
\begin{equation*}
    a(n++)=f(n,a(n))\text{ for all }n\in \mathbb{N},
\end{equation*}
and furthermore that this function is unique.
\begin{proof}
    First show inductively that for every $n\in\mathbb{N}$, there exists a unique function $a_N:\{n\in\mathbb{N}:n\leq N\}\to \mathbb{N}$ such that $a_N(0)=c$ and $a_N(n++)=f(n,a_N(n))$ - let this statement be $P(N)$. Assume $P(N)$ is true, we need to show that 
    $P(N++)$ is also true. Define $a_{N++}$ as follows: 
    \begin{equation*}
        \begin{gathered}
            \text{ if }n < N++, a_{N++}(n)=a_N(n),\\
            \text{ if }n = N++, a_{N++}(n)=f(N,a_N(N)).
        \end{gathered}
    \end{equation*}
    Then for the base case of $P(N++)$, $a_{N++}(0)=a_N(0)=c$ which satisfies the requirement. Now consider $a_{N++}(n++)$. Suppose $n<N++$. Then we have either $n++<N++$ or $n++=N++$. If $n++<N++$, by definition, we have $a_{N++}(n++)=a_N(n++)=f(n,a_N(n))$. Since $n<N++$,
    $a_{N++}(n)=a_N(n)$. Therefore, $a_{N++}(n++)=f(n,a_N(n))=f(n,a_{N++}(n))$. If $n++=N++$, $a_{N++}(n++)=a_{N++}(N++)=f(N,a_N(N))$. Since $n++=N++\iff n=N$, $a_{N++}=f(n,a_N(n))$. Since $n<N++$, $a_N(n)=a_{N++}(n)$. Therefore, $a_{N++}=f(n,a_{N++}(n))$. 
    Thus, $a_{N++}$ exists. 
    
    Define $a(n)=a_N(n), N=n$. Thus, $a(0)=a_0(0)=c$ and $a(n++)=a_{N++}(n++)=f(n,a_{N++}(n))=f(n,a_N(n))=f(n,a(n))$. Hence, there exists such a function $a$.

    Lastly, we want to show the uniqueness of $a$. Assume there exists another function $g$ such that $g(0)=c$ and $g(n++)=g(n,f(n))$ for all $n\in \mathbb{N}$. Then $g(0)=a(0)=c$. Assume $a(n)=g(n)$, $g(n++)=f(n,g(n))=f(n,a(n))=a(n++)$. Therefore, $a(n)=g(n)$ for all $n\in \mathbb{N}$. 
    Hence, $g=a$ and $a$ is unique.
\end{proof}
\subsubsection*{Exercise 3.5.13}
The purpose of this exercise is to show that there is essentially only one version of the natural number system in set theory. Suppose we have a set $\mathbb{N}'$ of ``alternative natural numbers'', an ``alternative zero'' 0', and 
an ``alternative increment operation'' which takes any alternative natural number $n'\in\mathbb{N}'$ and returns another alternative natural number $n'++'\in\mathbb{N}'$, such that the Peano axioms all hold with the natural numbers, zero, and increment replaced by their alternative counterparts.
Show that there exists a bijection $\mathbb{N}\to \mathbb{N}'$ from the natural numbers to the alternative natural numbers such that $f(0)=0'$, and such that for any $n\in\mathbb{N}$ and $n'\in\mathbb{N}'$, we have $f(n)=n'$ if and only if $f(n++)=n'++'$.
\begin{proof}
    Define $f$ as $f(0)=0'$ and $f(n++)=f(n)++'$. We need to show that $f(n)=n'\iff f(n++)=n'++'$.

    Suppose $f(n)=n'$. Then $f(n++)=f(n)++'=n'++'$. Suppose $f(n++)=n'++'$. By definition, $f(n++)=f(n)++'$. Then, $f(n)++'=n'++'$. So $f(n)=n'$. Hence, $f(n)=n'\iff f(n++)=n'++'$.

    Then we need to show that $f$ is bijective. Let $f(n)=n'$. Suppose $n_1$ and $n_2$ are two natural numbers. Assume $f(n_1)=f(n_2)$. By definition, $f(n++)=f(n)++'$, $f(n)=f(n-1)++',\dotsc$. $f(n_1)=f(n_2)$ only if the recursive steps reach the base case at the same time, in other words,
    $f(n_1)=f(n_2)$ only if $n_1=n_2$. Thus, $f$ is injective. 

    Use induction to show that $f$ is surjective. Consider the base case $0'$, by definition, $f(0)=0'$ so the base case is proved. Assume for $n'$, there exists $n\in \mathbb{N}$ such that $f(n)=n'$. Then 
    $n'++'=f(n)++'=f(n++)$, and $n++\in\mathbb{N}$. Thus, for every $n'\in\mathbb{N}'$, there exists $n\in\mathbb{N}$ such that $f(n)=n'$. Thus, $f$ is surjective.

    Since $f$ is both injective and surjective, it is bijective.
\end{proof}
\end{document}