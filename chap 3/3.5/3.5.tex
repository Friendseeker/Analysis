\documentclass[12pt, letter]{article}
\usepackage[utf8]{inputenc}
\usepackage[a4paper, total={6in, 8in}]{geometry}
\usepackage{tikz}
\usepackage[T1]{fontenc}
\usepackage{listings}
\usepackage{graphicx}
\usepackage{amsfonts}
\usepackage{amsmath}
\usepackage{amssymb}
\usepackage{amsthm}
\usepackage{mathtools}
\usepackage{listings}
\usepackage{bm}
\newcommand{\uvec}[1]{\boldsymbol{\hat{\textbf{#1}}}}
\usepackage[english]{babel}
\newtheorem{theorem}{Theorem}
\usepackage{setspace}

\setstretch{1.25}
\begin{document}
\section*{3.5 Cartesian products}
\subsubsection*{Definition 3.5.1 (Ordered pair).}
If $x$ and $y$ are any objects (possibly equal), we define the ordered pair $(x,y)$ to be a new object, consisting of $x$
as its first component and $y$ as its second component. Two ordered pairs $(x,y)$ and $(x',y')$ are considered equal if and only if 
both their components match, i.e.
\begin{equation*}
    (x,y)=(x',y')\iff (x=x'\text{ and }y=y').
\end{equation*}
\subsubsection*{Definition 3.5.4 (Cartesian product).}
If $X$ and $Y$ are sets, then we define the Cartesian product $X\times Y$ to be the collection of ordered pairs, whose first component
lies in $X$ and second component lies if $Y$, thus
\begin{equation*}
    X\times Y=\{(x,y):x\in X, y\in Y\}
\end{equation*}
or equivalently
\begin{equation*}
    a\in (X\times Y)\iff (a=(x,y) \text{ for some } x\in X and y\in Y).
\end{equation*}
\subsubsection*{Definition 3.5.7 (Ordered $n$-tuple and $n$-fold Cartesian product).}
Let $n$ be a natural number. An ordered $n$-tuple $(x_i)_{1\leq i\leq n}$ (also denoted $(x_1,\cdots, x_n)$) is a collection of objects $x_i$,
one for every natural number $i$ between $1$ and $n$; we refer to $x_i$ as the $i^{th}$ component of the ordered $n$-tuple. Two 
ordered $n$-tuples $(x_i)_{1\leq i\leq n}$ and $(y_i)_{1\leq i\leq n}$ are said to be equal iff $x_i=y_i$ for all $1\leq i\leq n$. 
If $(X_i)_{1\leq i\leq n}$ is an ordered $n$-tuple of sets, we define their Cartesian product $\prod_{1\leq i\leq n}X_i$ (also denoted 
$\prod_{i=1}^{n}X_i$ or $X_1\times \cdots \times X_n$) by
\begin{equation*}
    \prod_{i\leq i\leq n}X_i:=\{(x_i)_{1\leq i\leq n}:x_i\in X_i\text{ for all }1\leq i\leq n\}.
\end{equation*}
\subsubsection*{Lemma 3.5.12 (Finite choice).}
Let $n\geq 1$ be a natural number, and for each natural number $1\leq i\leq n$, let $X_i$ be a non-empty set. Then there exists an $n$-tuple $(x_i)_{1\leq i\leq n}$
such that $x_i\in X_i$ for all $1\leq i\leq n$. In other words, if each $X_i$ is non-empty, then the set $\prod_{1\leq i\leq n}X_i$
is also non-empty.
\subsection*{Exercises}
\subsubsection*{Exercise 3.5.1}
Suppose we define the ordered pair $(x,y)$ for any objects $x$ and $y$ by the formula $(x,y):=\{\{x\},\{x,y\}\}$ (thus using several applications of Axiom 3.4).
Thus for instance $(1,2)$ is the set $\{\{1\},\{1,2\}\}$, $(2,1)$ is the set $\{\{2\},\{2,1\}\}$, and $(1,1)$ is the set $\{1\}$.
Show that such a definition indeed obeys the property $(3.5)$, and also whenever $X$ and $Y$ are sets, the Cartesian product $X\times Y$ is also a set. 
Thus this definition can be validly used as a definition of an ordered pair. For an additional challenge, show that the alternate definition $(x,y):=\{x,\{x,y\}\}
$ also verifies $(3.5)$ and is thus also an acceptable definition of ordered pair.
\begin{enumerate}
    \item Show that $(x,y):=\{\{x\},\{x,y\}\}$ is a valid definition of an ordered pair.
    \begin{proof}
        First, we need to show that 
        \begin{equation*}
            (x,y)=(x',y')\iff (x=x'\text{ and }y=y').
        \end{equation*} 

        Suppose $(x,y)=(x',y')$. Then by definition, $\{\{x\},\{x,y\}\}=\{\{x'\},\{x',y'\}\}$. If $x=y$, $\{\{x\},\{x,y\}\}=\{\{x\},\{x\}\}=\{\{x\}\}$. 
        Then $\{\{x'\},\{x',y'\}\}$ must also only contain one element. Thus, $\{x'\}=\{x',y'\}$. So $x'=y'$. Lastly, we have $\{\{x\}\}=\{\{x'\}\}$. So $x=x'$. 
        Thus, $x=x'=y=y'$. It is the same thing if we assume $x'=y'$ at first. Now consider the case $x\ne y$. Then 
        $\{x\}$ has one element and $\{x,y\}$ has two elements. And since $\{x'\}$ could only contain one element, we have the following relations:
        \begin{equation*}
            \begin{cases}
                \{x\}=\{x'\}\implies x=x'\\
                \{x,y\}=\{x',y'\}\text{ and }x=x' \implies y=y'.
            \end{cases}
        \end{equation*}
        It would be the same if we assume $x'\ne y'$. Thus, $(x,y)=(x',y')\implies (x=x'\text{ and }y=y')$

        Suppose $x=x'$ and $y=y'$. Then we must have $\{x\}=\{x'\}$ and $\{x,y\}=\{x',y'\}$. Hence, $\{\{x\},\{x,y\}\}=\{\{x'\},\{x',y'\}\}$. 
        Thus, $(x=x'\text{ and }y=y')\implies (x,y)=(x',y')$.

        Therefore, $(x,y)=(x',y')\iff (x=x'\text{ and }y=y')$. This definition verifies $(3.5)$. 
    
        Then, we need to show that whenever $X$ and $Y$ are sets, the Cartesian product $X\times Y$ is a set.
        Use the definition above: $(x,y)=\{\{x\},\{x,y\}\}$. The powerset of $X\cup Y$ is $\{0,\{x\},\{y\},\{x,y\}\}$ which contains the elements in 
        $(x,y)$. Then the powerset of the powerset of $X\cup Y$ contains $(x,y)$. The elements in $\mathcal{P}(\mathcal{P}(X\cup Y))$ is a set, thus, the Cartesian 
        product is a set. More specifically, 
        \begin{equation*}
            X\times Y=\{z\in \mathcal{P}(\mathcal{P}(X\cup Y)):z \text{ contains exactly one singleton set and one pair set}\}.
        \end{equation*}
    \end{proof}
    \item Show that $(x,y)=\{x,\{x,y\}\}$ is also a valid definition of an ordered pair.
    \begin{proof}
        We need to show that 
    \begin{equation*}
        \{x,\{x,y\}\}=\{x',\{x',y'\}\}\iff x=x'\text{ and }y=y'.    
    \end{equation*}
    Suppose $\{x,\{x,y\}\}=\{x',\{x',y'\}\}$. Denote $A=\{x,y\}$, $B=\{x',y'\}$. Then $\{x,A\}=\{x',B\}$. Since sets are objects, $\{x,A\}$ and 
   $\{x',B\}$ are both pair sets. Since $x\in \{x,\{x,y\}\}$ and $x\in\{x,\{x,y\}\}\implies x\in\{x',\{x',y'\}\}$. So either $x=x'$ or $x=\{x',y'\}$.
   Assume $x=\{x',y'\}$. Then the only option left is $x'=\{x,y\}$. As $x$ and $x'$ are both sets, having $x\in x'$ and $x'\in x$ at the same time violates
   the statements in Exercise 3.2.2. Therefore, $x=x'$ and $\{x,y\}=\{x',y'\}=\{x,y'\}$. Thus, $y=y'$. 
   
   Suppose $x=x'$ and $y=y'$. Then clearly we have $\{x,y\}=\{x',y'\}$. So $\{x,\{x,y\}\}=\{x',\{x',y'\}\}$. 

   Thus, $(x,y):=\{x,\{x,y\}\}$ verifies $(3.5)$.
    \end{proof}
\end{enumerate}
\subsubsection*{Exercise 3.5.2}
Suppose we define an ordered $n$-tuple to be a surjective function $x:\{i\in \mathbb{N}:1\leq i\leq n= n\}\to X$ whose range is some arbitrary set $X$
(so different ordered $n$-tuples are allowed to have different ranges); we then write $x_i$ for $x(i)$, and also write $x$ as $(x_i)_{1\leq i\leq n}$.
Using this definition, verify that we have $(x_i)_{1\leq i\leq n}=(y_i)_{1\leq i\leq n}$ if and only if $x_i=y_i$ for all $1\leq i\leq n$. Also,
show that if $(X_i)_{1\leq i\leq n}$ is an ordered $n$-tuple of sets, then the Cartesian product, as defined in Definition 3.5.7, is indeed a set.
\begin{enumerate}
    \item $(x_i)_{1\leq i\leq n}=(y_i)_{1\leq i\leq n}\iff x_i=y_i$ for all $1\leq i\leq n$.
    \begin{proof}
        Apparently, $x$ and $y$ have the same domain $\{i\in\mathbb{N}:1\leq i\leq n\}$. Suppose $y:\{i\in \mathbb{N}:1\leq i\leq n\}\to Y$. 

        Suppose $(x_i)_{1\leq i\leq n}=(y_i)_{1\leq i\leq n}$. Since $x$ and $y$ are two functions, we must have $X$=$Y$ so that they have the same range.
        And by Definition 3.3.7, we have $x(i)=y(i)$ for all $1\leq i\leq n$. Therefore, $(x_i)_{1\leq i\leq n}=(y_i)_{1\leq i\leq n}\implies x_i=y_i$ for all $1\leq i\leq n$..
    
        Suppose $x_i=y_i$ for all $1\leq i\leq n$. Since $x$ and $y$ are both surjective and $\{x_1,\cdots, x_n\}=\{y_1,\cdots, y_n\}$, $X=Y=\{x_1,\cdots, x_n\}=\{y_1,\cdots, y_n\}$. 
        Thus, $x$ and $y$ have the same range. And because they also have the same domain and $x_i=y_i$ for all $1\leq i\leq n$, $x=y$. Therefore, we have proved $x_i=y_i$ for all $1\leq i\leq n
        \implies x=y$. 
        
        Thus, $(x_i)_{1\leq i\leq n}=(y_i)_{1\leq i\leq n}\iff x_i=y_i$ for all $1\leq i\leq n$.
    \end{proof} 
    \item If $(X_i)_{1\leq i\leq n}$ is an ordered $n$-tuple of sets, then the Cartesian product is a set.
    \begin{proof}
        Denote $A=\{X_1, X_2,\cdots, X_n\}$, so every element of $A$ is a set itself and by the union axiom we have $\bigcup A$ being the set consists of all the elements of the elements of $A$. 
        Denote $I=\{i\in \mathbb{N}: 1\leq i\leq n\}$. Then, the mapping function $x$ would be partial functions with domain $I$ which is also a subset of $I$ and range being a subset of $\bigcup A$. Denote it as $X$. 
        Thus, by Exercise 3.4.7, the collection of all these partial functions is a set. By Definition 3.5.7, the Cartesian product would be a subset of the set of all these partial functions.
        Let $P(x)$ be $x_i\in X_i$ for all $1\leq i\leq n$. By Axiom of specification, there exists a set $\{x\in X: P(x)\text{ is true}\}$ which is the same 
        as the Cartesian product. Therefore, the Cartesian product is indeed a set.
    \end{proof}
\end{enumerate}
\subsubsection*{Exercise 3.5.3}
Show that the definitions of equality for ordered pair and ordered $n$-tuple obey the reflexivity, symmetry, and transitivity axioms.
\begin{itemize}
    \item reflexivity
    \begin{proof}
        For the ordered pair $(x,y)$, since $x=x$ and $y=y$, we have $(x,y)=(x,y)$. 
        For the ordered $n$-tuple $(x_i)_{1\leq i\leq n}$, since $x_i=x_i$ for $1\leq i\leq n$, by definition, we have $(x_i)_{1\leq i\leq n}$.     
    \end{proof}  
    \item symmetry
    \begin{proof}
        We want to show $(x,y)=(x',y')\iff (x',y')=(x,y)$. Assume $(x,y)=(x',y')$. Then $x=x'$ and $y=y'$. By symmetry property of equality, we have 
        $x'=x$ and $y'=y$. By definition, we have $(x',y')=(x,y)$. Similarly, we can show that $(x',y')=(x,y)\implies (x,y)=(x',y')$.
        Thus, $(x,y)=(x',y')\iff (x',y')=(x,y)$.

        For ordered $n$-tuple, we want to show that $(x_i)_{1\leq i\leq n}=(y_i)_{1\leq i\leq n}\iff(y_i)_{1\leq i\leq n}= (x_i)_{1\leq i\leq n}$.
        Assume $(x_i)_{1\leq i\leq n}=(y_i)_{1\leq i\leq n}$. Then $x_i=y_i$ for every $1\leq i\leq n$. By the symmetry property of equality, we have 
        $y_i=x_i$ for every $1\leq i\leq n$. Therefore, by definition of ordered $n$-tuple, we have $(y_i)_{1\leq i\leq n}= (x_i)_{1\leq i\leq n}$. 
        The approach is the same for the other way around. Thus, $(x_i)_{1\leq i\leq n}=(y_i)_{1\leq i\leq n}\iff(y_i)_{1\leq i\leq n}= (x_i)_{1\leq i\leq n}$.
    \end{proof}  
    \item transitivity 
    \begin{proof}
        The proof for ordered pair is omitted since it is only a special case of ordered $n-$tuple. 
        We need to show that $(x_i)_{1\leq i\leq n}=(y_i)_{1\leq i\leq n}\text{ and } (y_i)_{1\leq i\leq n}=(z_i)_{1\leq i\leq n} \implies 
        (x_i)_{1\leq i\leq n}=(z_i)_{1\leq i\leq n}$. Since $(x_i)_{1\leq i\leq n}=(y_i)_{1\leq i\leq n}$, we have $x_i=y_i$ for $1\leq i\leq n$. 
        Since $(y_i)_{1\leq i\leq n}=(z_i)_{1\leq i\leq n}$, we have $y_i=z_i$ for $1\leq i\leq n$. By transitivity property of equality, we have $x_i=z_i$. 
        By definition of ordered $n$-tuple, $(x_i)_{1\leq i\leq n}=(z_i)_{1\leq i\leq n}$. 
    \end{proof}
\end{itemize}
\subsubsection*{Exercise 3.5.4}
Let $A$, $B$, $C$ be sets. Show that $A\times (B\cup C)=(A\times B)\cup (A\times C)$, that $A\times (B\cap C)=(A\times B)\cap (A\times C)$, and 
that $A\times (B\backslash C)=(A\times B)\backslash(A\times C)$.
\begin{enumerate}
    \item $A\times (B\cup C)=(A\times B)\cup (A\times C)$.
    \begin{proof}
        We need to show that $(x,y)\in A\times (B\cup C)\iff (x,y)\in (A\times B)\cup (A\times C)$.

        Suppose $(x,y)\in A\times (B\cup C)$. By definition, we have $x\in A$ and $y\in (B\cup C)$. $y\in B\cup C\iff y\in B\text{ or }y\in C$. 
        Therefore, $(x,y)\in (A\times B)$ or $(x,y)\in (A\times C)$. Hence, $(x,y)\in (A\times B)\cup (A\times C)$. 

        Suppose $(x,y)\in (A\times B)\cup (A\times C)$. Then we have either $(x,y)\in A\times B$ or $(x,y)\in A\times C$. 
        $(x,y)\in A\times B\implies x\in A\text{ and }y\in B$. $(x,y)\in A\times C\implies x\in A\text{ and }y\in C$. Therefore, we have $x\in A$ and $y\in B\cup C$. 
        Hence, $(x,y)\in A\times (B\cup C)$. 

        Thus, $A\times (B\cup C)=(A\times B)\cup (A\times C)$.
    \end{proof}
    \item $A\times (B\cap C)=(A\times B)\cap (A\times C)$.
    \begin{proof}
        We need to show that $(x,y)\in A\times (B\cap C)\iff (x,y)\in (A\times B)\cap (A\times C)$.

        Suppose $(x,y)\in A\times (B\cap C)$. Then $x\in A$ and $y\in B\cap C$. $y\in B\cap C\iff y\in B\text{ and }y\in C$. Then we have $(x,y)\in A\times B$ and 
        $(x,y)\in A\times C$. Hence, $(x,y)\in (A\times B)\cap (A\times C)$. 

        Suppose $(x,y)\in (A\times B)\cap (A\times C)$. Then $(x,y)\in A\times B$ and $(x,y)\in A\times C$. $(x,y)\in A\times B\implies x\in A$ and $y\in B$. 
        $(x,y)\in A\times C\implies x\in A$ and $y\in C$. Overall, we have $x\in A$ and $y\in B\cap C$. Hence, $(x,y)\in A\times (B\cap C)$.
        
        Thus, $A\times (B\cap C)=(A\times B)\cap (A\times C)$.
    \end{proof}
    \item $A\times (B\backslash C)=(A\times B)\backslash (A\times C)$.
    \begin{proof}
        We need to show that $(x,y)\in A\times (B\backslash C)\iff (x,y)\in (A\times B)\backslash (A\times C)$.

        Suppose $(x,y)\in A\times (B\backslash C)$. Then $x\in A$ and $y\in B$ and $y\notin C$. $x\in A$ and $y\in B\implies (x,y)\in A\times B$.
        $y\notin C\implies (x,y)\notin A\times C$. Therefore, $(x,y)\in (A\times B)\backslash (A\times C)$.

        Suppose $(x,y)\in (A\times B)\backslash (A\times C)$. Then $(x,y)\in A\times B$ and $(x,y)\notin A\times C$. $(x,y)\in A\times B\implies x\in A$ and $y\in B$. 
        $(x,y)\notin A\times C$ and $x\in A\implies y\notin C$. Therefore, we have $x\in A$ and $y\in B$ and $y\notin C$. Hence, $(x,y)\in A\times (B\backslash C)$.

        Thus, $A\times (B\backslash C)=(A\times B)\backslash (A\times C)$.
    \end{proof}
\end{enumerate}
\subsubsection*{Exercise 3.5.5}
Let $A$, $B$, $C$, $D$ be sets. Show that $(A\times B)\cap (C\times D)=(A\cap C)\times (B\cap D)$. Is it true that $(A\times B)\cup (C\times D)=(A\cup C)\times (B\cup D)$?
Is it true that $(A\times B)\backslash (C\times D)=(A\backslash C)\times (B\backslash D)$?
\begin{proof}
    $(A\times B)\cap (C\times D)=(A\cap C)\times (B\cap D)\iff ((x,y)\in(A\times B)\cap (C\times D)\iff (x,y)\in (A\cap C)\times (B\cap D))$.
    Suppose $x\in (A\times B)\cap (C\times D)$. $(x,y)\in A\times B\implies x\in A$ and $y\in B$. $(x,y)\in C\times D\implies x\in C$ and $y\in D$. Therefore, we have
    $x\in A\cap C$ and $y\in B\times D$. Thus, $(x,y)\in (A\cap C)\times (B\cap D)$.
    Suppose $(x,y)\in (A\cap C)\times (B\cap D)$. Then $x\in A\cap C$ and $y\in B\cap D$. $x\in A\cap C\implies x\in A$ and $x\in C$. $y\in B\cap D\implies y\in B$ and $y\in D$.
    $x\in A$ and $y\in B\implies (x,y)\in A\times B$. $x\in C$ and $y\in D\implies (x,y)\in C\times D$. Hence, $(x,y)\in (A\times B)\cap (C\times D)$. 
    Thus, $(A\times B)\cap (C\times D)=(A\cap C)\times (B\cap D)$. 

    $(A\times B)\cup (C\times D)=(A\cup C)\times (B\cup D)$ is not true. Assume $(x,y)\in (A\cup C)\times (B\cup D)$. And suppose $x\in A$ and $y\in D$.
    Then $(x,y)\in A\times D$ and $(x,y)\notin A\times B$ and $(x,y)\notin C\times D$. Hence, $x\notin (A\times B)\cup (C\times D)$. Thus, $(A\times B)\cup (C\times D)=(A\cup C)\times (B\cup D)$ is not true.

    $(A\times B)\backslash (C\times D)=(A\backslash C)\times (B\backslash D)$ is not true. A counterexample: $x\in A\cap C$ and $y\in B\backslash D$. Then 
    $(x,y)\in(A\times B)\backslash (C\times D)$ but $(x,y)\notin (A\backslash C)\times (B\backslash D)$. Thus, $(A\times B)\backslash (C\times D)=(A\backslash C)\times (B\backslash D)$ is not true.
\end{proof}
\subsubsection*{Exercise 3.5.6}
Let $A,B,C,D$ be non-empty sets. Show that $A\times B\subseteq C\times D$ if and only if $A\subseteq C$ and $B\subseteq D$, and that $A\times B=C\times D$ if and only if $A=C$ and $B=D$. 
What happens if the hypotheses that the $A,B,C,D$ are all non-empty are removed?
\begin{enumerate}
    \item $A\times B\subseteq C\times D\iff A\subseteq C$ and $B\subseteq D$.
    \begin{proof}
        Suppose $A\times B\subseteq C\times D$, that is $(x,y)\in A\times B\implies (x,y)\in C\times D$. Since $A,B,C,D$ are non-empty, we have two conditions: $x\in A\implies x\in C$ and
        $y\in B\implies y\in D$. Thus, $A\subseteq C$ and $B\subseteq D$.

        Suppose $A\subseteq C$ and $B\subseteq D$. Then we have $x\in A\implies x\in C$ and $y\in B\implies y\in D$. Combining these two conditions, $(x,y)\in A\times B\implies (x,y)\in C\times D$.
        Hnece, $A\times B\subseteq C\times D$.

        Thus, $A\times B\subseteq C\times D\iff A\subseteq C$ and $B\subseteq D$.
    \end{proof}
    \item $A\times B=C\times D\iff A=C$ and $B=D$.
    \begin{proof}
        Suppose $A\times B=C\times D$. Then $(x,y)\in A\times B\iff (x,y)\in C\times D$. Since $A,B,C,D$ are non-empty, we have $x\in A\iff x\in C$ and 
        $y\in B\iff y\in D$. Therefore, $A=C$ and $B=D$.

        Suppose $A=C$ and $B=D$. Then $x\in A\iff x\in C$ and $y\in B\iff y\in D$. Therefore, $(x,y)\in A\times B\iff (x,y)\in C\times D$. Hence, $A\times B=C\times D$.

        Thus, $A\times B=C\times D\iff A=C$ and $B=D$.
    \end{proof}
\end{enumerate}
If the hypothese that the $A,B,C,D$ are all non-empty are removed, the equalities will not hold any more. A counterexample would be $A$ is non-empty, $B=\emptyset$, $C=\emptyset$, and $D$ is non-empty. 
$A\times B\subseteq C\times D$ but $A$ is not a subset of $C$. $A\times B=C\times D$ but $A\ne C$.
\end{document}