\documentclass[12pt, letter]{article}
\usepackage[utf8]{inputenc}
\usepackage[a4paper, total={6in, 8in}]{geometry}
\usepackage{tikz}
\usepackage[T1]{fontenc}
\usepackage{listings}
\usepackage{graphicx}
\usepackage{amsfonts}
\usepackage{amsmath}
\usepackage{amssymb}
\usepackage{amsthm}
\usepackage{mathtools}
\usepackage{listings}
\usepackage{bm}
\newcommand{\uvec}[1]{\boldsymbol{\hat{\textbf{#1}}}}
\usepackage[english]{babel}
\newtheorem{theorem}{Theorem}
\usepackage{setspace}

\setstretch{1.25}
\begin{document}
\section*{3.2 Russell's Paradox}
\subsubsection*{Axiom 3.9 (Universal spcification).}
(Dangerous!) Suppose for every object $x$ we have a property $P(x)$ pertaining to $x$ (so that for every $x$, $P(x)$ is either a true statement of false statement). 
Then there exists a set $\{x:P(x)\text{ is true}\}$ such that for every object $y$,
\begin{equation*}
    y\in\{x:P(x)\text{ is true}\} \iff P(y) \text{ is true}.
\end{equation*}
\subsubsection*{Axiom 3.10 (Regularity).}
If $A$ is a non-empty set, then there is at least one element $x$ of $A$ which is either not as set, or is disjoint from $A$.

\subsection*{Exercises}
\subsubsection*{Exercise 3.2.1}
Show that the universal specification axiom, Axiom 3.9, if assumed to be true, would imply Axioms 3.3, 3.4, 3.5, 3.6, and 3.7. Thus, this axiom, if 
permitted, would simplify the foundations of set theory tremendously. Unfortunately, as we have een, Axiom 3.9 is ``too good to be true''!
\begin{proof}
    Basically, we want to give a list of statements so that we could construct the sets as desired.
    \begin{enumerate}
        \item Axiom 3.3 
        $P(x)\iff x\ne x$.
        \item Axiom 3.4
        Singleton set: $P(x)\iff x=a$.
        Pair set: $P(x)\iff x=a \text{ or } x=b$.
        \item Axiom 3.5
        $P(x)\iff x\in A\text{ or }x\in B$.
        \item Axiom 3.6
        $P_1(x)\iff x\in A\text{ and }P(x)\text{ is true}$.
        \item Axiom 3.7 
        $P(y)\iff P(x,y)\text{ is true for some }x\in A$.
    \end{enumerate}
\end{proof}
\subsubsection*{Exercise 3.2.2}
Use the axiom of regularity (and the singleton set axiom) to show that if $A$ is a set, then $A\notin A$. Furthermore, show that 
if $A$ and $B$ are two sets, then either $A\notin B$ or $B\notin A$ (or both).
\begin{proof}
    Suppose $A$ is a set and $A\in A$. We have $A$ is a set and $A$ is not disjoint from $A$ which is a contradiction. Thus, $A\notin A$.

    Suppose $A$ and $B$ are two sets, $A\in B$ and $B\in A$. Since $A\in B$ and $A$ is a set, by Axiom 3.10, we have $A$ is disjoint from $B$.
    Thus, $B\notin A$ must stand. (contradiction) Then we need to show that $A\notin B$, $B\notin A$, $A\notin B$ and $B\notin A$ are possible by giving examples.
    Let $A=\{1\}, B=\{1,3\}$, then $B\notin A$. Let $A=\{1,3\}$, $B=\{1\}$, then $A\notin B$. Let $A={1}$, $B=\{2\}$, then $A\notin B$ and $B\notin A$. Thus, we have shown that 
    either $A\notin B$ or $B\notin A$ (or both).  
\end{proof}
\subsubsection*{Exercise 3.2.3}
Show (assuming the other axioms of set theory) that the universal specification axiom, Axiom 3.9, is equivalent to an axiom postulating the existence of a "universal set" $\Omega$ consisting of 
all objects (i.e., for all objects $x$, we have $x\in\Omega$). In other words, if Axiom 3.9 is true, then a universal set exists, and conversely, if a universal set exists, then Axiom 3.9 is true. 
(This may explain why Axiom 3.9 is called the axiom of universal specification). Note that if a universal set $\Omega$ existed, then we would have $\Omega\in\Omega$ by Axiom 3.1, contradicting Exercise 3.2.2. Thus 
the axiom of foundation specifically rules out the axiom of universal specification.
\begin{proof}
    Let $P(x)$ be $x$ is an object. Therefore, $\Omega$ consists of all objects $x$. (Axiom 3.9 $\Rightarrow$ existence of a universal set) On the other hand, 
    if $\Omega$ consists of all objects, then for any statement $P(x)$, all $x$ such that $P(x)$ is true must be in $\Omega$. (existence of a universal set $\Rightarrow$ Axiom 3.9) Thus, 
    Axiom 3.9 is equivalent to the existence of a ``universal set''. This cannot be possible since if $\Omega$ is a universal set, 
    $\Omega\in\Omega$. (contradiction)
\end{proof}
\end{document}