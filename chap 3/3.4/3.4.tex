\documentclass[12pt, letter]{article}
\usepackage[utf8]{inputenc}
\usepackage[a4paper, total={6in, 8in}]{geometry}
\usepackage{tikz}
\usepackage[T1]{fontenc}
\usepackage{listings}
\usepackage{graphicx}
\usepackage{amsfonts}
\usepackage{amsmath}
\usepackage{amssymb}
\usepackage{amsthm}
\usepackage{mathtools}
\usepackage{listings}
\usepackage{bm}
\newcommand{\uvec}[1]{\boldsymbol{\hat{\textbf{#1}}}}
\usepackage[english]{babel}
\newtheorem{theorem}{Theorem}
\usepackage{setspace}

\setstretch{1.25}
\begin{document}
\section*{3.4 Images and inverse images}
\subsubsection*{Definition 3.4.1 (Images of sets).}
If $f:X\to Y$ is a function from $X$ to $Y$, and $S$ is a set in $X$, we define $f(S)$ to be the set
\begin{equation*}
    f(S):=\{f(x):x\in S\};
\end{equation*}
this set is a subset of $Y$, and is sometimes called the image of $S$ under the map $f$. We sometimes call $f(S)$ the forward image of $S$
to distinguish it from the concept of the inverse image $f^{-1}(S)$ of $S$, which is defined below.

\subsubsection*{Definition 3.4.5 (Inverse images).}
If $U$ is a subset of $Y$, we define the set $f^{-1}(U)$ to be the set 
\begin{equation*}
    f^{-1}(U):=\{x\in X:f(x)\in U\}.
\end{equation*}
In other words, $f^{-1}(U)$ consists of all the elements of $X$ which map into $U$:
\begin{equation*}
    f(x)\in U\iff x\in f^{-1}(U).
\end{equation*}
We feel $f^{-1}(U)$ the inverse image of $U$. 
\subsubsection*{Axiom 3.11 (Power set axiom).}
Let $X$ and $Y$ be sets. Then there exists a set, denoted $Y^X$, which consists of all the functions from $X$ to $Y$, thus
\begin{equation*}
    f\in Y^X\iff (f\text{ is a function with domain }X\text{ and range }Y).
\end{equation*}
\subsubsection*{Lemma 3.4.10}
Let $X$ be a set. Then the set 
\begin{equation*}
    \{Y:Y\text{ is a subset of }X\}
\end{equation*}
is a set.
\subsubsection*{Axiom 3.12 (Union).}
Let $A$ be a set, all of whose elements are themselves sets. Then there exists a set $\bigcup A$ whose elements are precisely those objects
which are elements of the elements of $A$, thus for all objects $x$
\begin{equation*}
    x\in\bigcup A\iff (x\in S\text{ for some }S\in A).
\end{equation*}
\subsection*{Exercises}
\subsubsection*{Exercise 3.4.1}
Let $f:X\to Y$ be a bijective function, and let $f^{-1}:Y\to X$ be its inverse. Let $V$ be any subset of $Y$.
Prove that the forward image of $V$ under $f^{-1}$ is the same set as the inverse image of $V$ under $f$; thus the fact that 
both sets are denoted by $f^{-1}(V)$ will not lead to any inconsistency.
\begin{proof}
    Let $U$ be the forward image of $V$ under $f^{-1}$,
    \begin{equation*}
        U=\{f^{-1}(y):y\in V\}.
    \end{equation*}
    And let $W$ be the inverse image of $V$ under $f$,
    \begin{equation*}
        W=\{x\in X: f(x)\in V\}.
    \end{equation*}
    We need to show that $U=W$ which can be done by proving $x\in U\iff x\in W$. 

    First, consider an arbitrary $x\in U$. Since the range of $f^{-1}$ is $X$, $x\in X$. And there exists exactly one $y\in V$ such that $x=f^{-1}(y)$. 
    By definition of inverse, we have $f(x)=y\in V$. Therefore, $x\in W$.

    Then, consider an arbitrary $x\in W$. Denote $y=f(x)$. Then we have $x\in X$ and $y=f(x)\in Y$. By definition, $x=f^{-1}(y)$. Therefore, $x\in U$.
    
    Thus, $x\in V\iff x\in U$. The statement has been proved. 
\end{proof}
\subsubsection*{Exercise 3.4.2}
Let $f:X\to Y$ be a function from one set $X$ to another set $Y$, let $S$ be a subset of $X$, and let $U$ be a subset of $Y$.
What, in general, can one say about $f^{-1}(f(S))$ and $S$? What about $f(f^{-1}(U))$ and $U$?
\begin{enumerate}
    \item $S\subseteq f^{-1}(f(S))$.
    \begin{proof}
        We need to show that $x\in S\implies x\in f^{-1}(f(S))$. Consider an arbitrary $x\in S$. Then $f(x)\in f(S)$. So $x=f^{-1}(f(x))\in f^{-1}(f(S))$. 
        $f^{-1}(f(S))\subseteq S$ does not stand, see p.58 for a counterexample. Thus, in general, we have $S\subseteq f^{-1}(f(S))$.
    \end{proof}
    \item $f(f^{-1}(U))\subseteq U$.
    \begin{proof}
        We need to show that $y\in f(f^{-1}(U))\implies y\in U$. Consider an arbitrary $y\in f(f^{-1}(U))$. Then there exists $x\in f^{-1}(U)$ such that $f(x)=y$. 
        Since $x\in f^{-1}(U)$, by definition of inverse images, $f(x)=y\in U$. $U\subseteq f(f^{-1}(U))$ is not true, see p.58 for a counterexample. Thus, in general, we have
        $f(f^{-1}(U))\subseteq U$.
    \end{proof}
\end{enumerate}
If $f$ is bijective, we have $S=f^{-1}(f(S))$ and $f(f^{-1}(U))=U$.
\subsubsection*{Exercise 3.4.3}
Let $A,B$ be two subsets of a set $X$, and let $f:X\to Y$ be a function. Show that $f(A\cap B)\subseteq f(A)\cap f(B)$, that $f(A)\backslash f(B)\subseteq f(A\backslash B)$, $f(A\cup B)=f(A)\cup f(B)$. For the first two statements, is it true 
that the $\subseteq$ relation can be improved to $=$?
\begin{enumerate}
    \item $f(A\cap B)\subseteq f(A)\cap f(B)$.
    \begin{proof}
        We need to show that $y\in f(A\cap B)\implies y\in f(A)\cap f(B)$. Assume $y\in f(A\cap B)$, then there exists $x\in A\cap B$ such that $y=f(x)$. $x\in A\cap B\iff (x\in A)\land(x\in B)$. $x\in A\implies y=f(x)\in f(A)$, $x\in B\implies y=f(x)\in f(B)$. 
        So $(y\in f(A))\land(y\in f(B))$. Therefore, $y\in f(A)\cap f(B)$.

        The $\subseteq$ relation cannot be improved to $=$. A counterexample: $A:\{0,1\}$, $B:\{1,2\}$, $f(0)=2$, $f(1)=1$, $f(2)=2$.
    \end{proof}
    \item $f(A)\backslash f(B)\subseteq f(A\backslash B)$.
    \begin{proof}
        We need to show that $y\in f(A)\backslash f(B)\implies y\in f(A\backslash B)$. Assume $y\in f(A)\backslash f(B)$ which means $y\in f(A)\land y\notin f(B)$. Since $y\in f(A)$, there exists $x\in A$ such that $f(x)=y$. 
        On the other hand, $y\notin f(B)$ so $x\notin B$ (otherwise we will have $y=f(x)\in B$). So there exists $(x\in A)\land (x\notin B) \iff x\in (A\backslash B)$ such that $y=f(x)$. Thus, $y\in A\backslash B$.

        The $\subseteq$ relation cannot be improved to $=$. A counterexample: $A:\{1,2\}$, $B:\{2\}$, $f(1)=1$, $f(2)=1$.
    \end{proof}
    \item $f(A\cup B)=f(A)\cup f(B)$.
    \begin{proof}
        We need to show that $y\in f(A\cup B)\iff y\in f(A)\cup f(B)$. 
        
        First, suppose $y\in f(A\cup B)$. Then there exists $x\in A\cup B$ such that $y=f(x)$. $x\in A\cup B\implies (x\in A)\lor (x\in B)$. If $x\in A$, since $y=f(x)$, $y\in f(A)$. If $x\in B$, since $y=f(x)$, $y\in f(B)$.
        So $y\in f(A)$ or $y\in f(B)$. Thus, $y\in f(A)\cup f(B)$. 

        Then, suppose $y\in f(A)\cup f(B)$. If $y\in f(A)$, $\exists x\in A$ such that $y=f(x)$. $x\in A\implies x\in A\cup B$. So $y\in f(A\cup B)$. Similarly, if $y\in f(B)$, we also conclude that $y\in f(A\cup B)$. Therefore, in both cases, we have $y\in f(A\cup B)$.
        Thus, $f(A\cup B)=f(A)\cup f(B)$.
    \end{proof}
\end{enumerate}
\subsubsection*{Exercise 3.4.4}
Let $f:X\to Y$ be a function from one set $X$ to another set $Y$, and let $U,V$ be subsets of $Y$. Show that $f^{-1}(U\cup V)=f^{-1}(U)\cup f^{-1}(V)$, that $f^{-1}(U\cap V)=f^{-1}(U)\cap f^{-1}(V)$, and that $f^{-1}(U\backslash V)=f^{-1}(U)\backslash f^{-1}(V)$.
\begin{enumerate}
    \item $f^{-1}(U\cup V)=f^{-1}(U)\cup f^{-1}(V)$.
    \begin{proof}
        We need to show that $x\in f^{-1}(U\cup V)\iff x\in f^{-1}(U)\cup f^{-1}(V)$. 

        First, suppose $x\in f^{-1}(U\cup V)$. Then there exists $y\in U\cup V$ such that $f(x)=y$. If $y\in U$, $x\in f^{-1}(U)$. If $y\in V$, $x\in f^{-1}(V)$. So $x\in f^{-1}(U)$ or $x\in f^{-1}(V)$. Thus, $x\in f^{-1}(U)\cup f^{-1}(V)$.

        Then, suppose $x\in f^{-1}(U)\cup f^{-1}(V)$ which means $x\in f^{-1}(U)$ or $x\in f^{-1}(V)$. If $x\in f^{-1}(U)$, then $\exists y\in U$ such that $y=f(x)$. If $x\in f^{-1}(V)$, then $\exists y\in V$ such that $y=f(x)$. So $y=f(x)\in U$ or $y=f(x)\in V$. So $y=f(x)\in U\cup V$.
        Thus, $x\in f^{-1}(U\cup V)$. 
        
        Thus, we have shown that $f^{-1}(U\cup V)=f^{-1}(U)\cup f^{-1}(V)$.
    \end{proof}
    \item $f^{-1}(U\cap V)=f^{-1}(U)\cap f^{-1}(V)$.
    \begin{proof}
        We need to show that $x\in f^{-1}(U\cap V)\iff x\in f^{-1}(U)\cap f^{-1}(V)$. 

        First, suppose $x\in f^{-1}(U\cup V)$. Then $\exists y\in U\cap V$ such that $y=f(x)$. Since $y\in U$, $x\in f^{-1}(U)$. Since $y\in V$, $x\in f^{-1}(V)$. And because $x\in f^{-1}(U)$ and $x\in f^{-1}(V)$, $x\in f^{-1}(U)\cap f^{-1}(V)$.

        Then, suppose $x\in f^{-1}(U)\cap f^{-1}(V)$. Then there exists $y=f(x)$ such that $y=f(x)$, $y\in U$ and $y\in V$. So $y=f(x)\in U\cap V$. Thus, $x\in f^{-1}(U\cap V)$.

        Thus, $f^{-1}(U\cap V)=f^{-1}(U)\cap f^{-1}(V)$.
    \end{proof}
    \item $f^{-1}(U\backslash V)=f^{-1}(U)\backslash f^{-1}(V)$.
    \begin{proof}
        We need to show that $x\in f^{-1}(U\backslash V)\iff x\in f^{-1}(U)\backslash f^{-1}(V)$. 
        First, suppose $x\in f^{-1}(U\backslash V)$. Then there exists $(y\in U)\land y\notin V$ such that $f(x)=y$. $y\in U\implies x\in f^{-1}(U)$. On the other hand, $x\notin f^{-1}(V)$ (otherwise $y=f(x)\in V$). 
        So $(x\in f^{-1}(U)) \land (x\notin f^{-1}(V))$. Hence, $x\in f^{-1}(U)\backslash f^{-1}(V)$.

        Then, suppose $x\in f^{-1}(U)\backslash f^{-1}(V)$ which means $x\in f^{-1}(U) \land x\notin f^{-1}(V)$. Since $x\in f^{-1}(U)$, there exists $y\in U$ such that $y=f(x)$. And since $x\notin f^{-1}(V)$, we must have $y\notin V$.
        So there exists $y\in U \land y\notin V \iff y\in U\backslash V$ such that $f(x)=y$. Hence, $x\in f^{-1}(U\backslash V)$.

        Thus, $f^{-1}(U\backslash V)=f^{-1}(U)\backslash f^{-1}(V)$.
    \end{proof}
\end{enumerate}
\subsubsection*{Exercise 3.4.5}
Let $f:X\to Y$ be a function from one set $X$ to another set $Y$. Show that $f(f^{-1}(S))=S$ for every $S\subseteq Y$ if and only if $f$ is surjective. 
Show that $f^{-1}(f(S))=S$ for every $S\subseteq X$ if and only if $f$ is injective.
\begin{enumerate}
    \item $f(f^{-1}(S))=S$ for every $S\subseteq Y$ if and only if $f$ is surjective. 
    \begin{proof}
        We need to show that $y\in f(f^{-1}(S))=S\iff f$ is surjective. And for the LHS, we have proved in 3.4.2 that $f(f^{-1}(S))\subseteq S$ not matter what kind of function $f$ is. So it would be sufficient to show that $S\subseteq f(f^{-1}(S))$. 
        
        First, suppose $f$ is surjective. We want to show that $y\in S\implies y\in f(f^{-1}(S))$. Since $f$ is surjective and $S\in Y$, there must exist $x\in X$ such that $f(x)=y$. Because $y=f(x)$ and $y\in S$, $x\in f^{-1}(S)$. Since $x\in f^{-1}(S)$ 
        and $y=f(x)$, $y\in f(f^{-1}(S))$. 

        Then, suppose $y\in S\implies y\in f(f^{-1}(S))$. We want to show that $f$ is surjective. Assume $f$ is not surjective. Then there exists $y$ and $S\subseteq Y$, such that $y\in S$ and $\forall x\in X$, $f(x)\ne y$. Since $f^{-1}(S)$ is a subset of $X$, 
        for all objects $x\in f^{-1}(S)$, $f(x)\ne y$. Thus, $y\notin f(f^{-1}(S))$, contradiction. Thus, $f$ is surjective.
        
        Thus, $f(f^{-1}(S))=S$ for every $S\subseteq Y$ if and only if $f$ is surjective. 
    \end{proof}
    \item $f^{-1}(f(S))=S$ for every $S\subseteq X$ if and only if $f$ is injective.
    \begin{proof}
        We need to show that $f^{-1}(f(S))=S\iff f$ is injective. For the LHS, it is not necessary to show that $S\subseteq f^{-1}(f(S))$ since we have proved in 3.4.2 that it stands generally. 
        So we only need to show $f^{-1}(f(S))\subseteq S$ for every $S\subseteq X\iff f$ is injective.
        
        First, suppose $f$ is injective. Assume $x\in f^{-1}(f(S))$. Then there exists $y\in f(S)$ such that $y=f(x)$. Since $y\in f(S)$, there exists $x'\in S$ such that $y=f(x')$. 
        And because $f$ is injective, $x=x'$. Therefore, $x\in S$.

        Next, suppose $x\in f^{-1}(f(S))\implies x\in S$. Assume $f$ is not injective. Then $\exists x,x'\in X, x\ne x'$ and $f(x)=f(x')=y$. 
        Let $S$ be $\{x'\}$. In this case, $y\in f(S)$ and $x\in f^{-1}(f(S))$. But $x\notin S$, contradiction. Hence, $f$ is injective.

        Thus, $f^{-1}(f(S))=S$ for every $S\subseteq X$ if and only if $f$ is injective.
    \end{proof}
\end{enumerate}
\end{document}