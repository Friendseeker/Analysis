\documentclass[12pt, letter]{article}
\usepackage[utf8]{inputenc}
\usepackage[a4paper, total={6in, 8in}]{geometry}
\usepackage{tikz}
\usepackage[T1]{fontenc}
\usepackage{listings}
\usepackage{graphicx}
\usepackage{amsfonts}
\usepackage{amsmath}
\usepackage{amssymb}
\usepackage{amsthm}
\usepackage{mathtools}
\usepackage{listings}
\usepackage{bm}
\newcommand{\uvec}[1]{\boldsymbol{\hat{\textbf{#1}}}}
\usepackage[english]{babel}
\newtheorem{theorem}{Theorem}
\usepackage{setspace}

\setstretch{1.25}
\begin{document}

\section*{2.2}
\subsection*{2.2.1}
For any natural numbers $a, b, c$, we have $(a+b)+c=a+(b+c)$.
\begin{proof}
    Induct on $b$ by keeping $a$ and $c$ fixed. Consider the base case $b=0$. In this case, LHS$=(a+0)+c=a+c$ and RHS$=a+(0+c)=a+c$. Now suppose that 
    $(a+b)+c=a+(b+c)$. We need to show that $(a+(b++))+c=a+((b++)+c)$: 
    \begin{equation*}
        \begin{aligned}
            \text{LHS}=(a+(b++))+c=((a+b)++)+c=(a+b+c)++,\\
            \text{RHS}=a+((b++)+c)=a+((b+c)++)=(a+b+c)++.
        \end{aligned}
    \end{equation*}
    Thus both sides are equal to each other, and we have closed the induction.
\end{proof}

\subsection*{2.2.2}
Let $a$ be a positive number. Then there exists exactly one natural number $b$ such that $b++=a$. (I'm assuming that it meant $a$ is a positive natural number.)
\begin{proof}
    Induct on $a$. Since $0$ is not positive, we consider the base case $a=1$. We have $b++=b+1=0+1=1$. Cancellation law tells us that $b=0$, which is unique. Now suppose that there exists exactly one natural number $b_0$ such that $b_0++=a$, we need to show that
    there exists exactly one natural number $b$ such that $b++=a++$. By Cancellation law, we have $b=a=b_0++$. Since $b$ is the successor of $b_0$ and $b_0$ is unique, $b$ is also unique. Thus we have closed the induction.
\end{proof}

\subsection*{2.2.3}
\subsubsection*{(a)}
$a\geq a$.
\begin{proof}
    There exists a natural number $0$ such that $a+0=a$. Thus, $a\geq a$.
\end{proof}
\subsubsection*{(b)}
If $a\geq b$ and $b\geq c$, then $a\geq c$.
\begin{proof}
    Since $a\geq b$, there exists a natural number $m$ such that $b+m=a$. Since $b\geq c$, there exists a natural 
    number $n$ such that $c+n=b$. Then $c+(n+m)=(c+n)+m=b+m=a$. Therefore, $c\geq a$.

    Thus, if $a\geq b$ and $b\geq c$, then $a\geq c$.
\end{proof}
\subsubsection*{(c)}
If $a\geq b$ and $b\geq a$, then $a=b$.
\begin{proof}
    Since $a\geq b$, there exists a natural number $m$ such that $b+m=a$. Since $b\geq a$, there exists a natural number $n$ such that $a+n=b$. Then we have $a+n=(b+m)+n=b+(m+n)=b$. 
    By Cancellation law, we have $m+n=0$ which leads to $m=0$, $n=0$. Thus, $a=a+0=b$. 

    Thus, if $a\geq b$ and $b\geq a$, then $a=b$.
\end{proof}
\subsubsection*{(d)}
$a\geq b$ if and only if $a+c\geq b+c$.
\begin{proof}
    First, we need to show that $a\geq b \Rightarrow a+c\geq b+c$. Since $a\geq b$, there exists a natural number $n$ such that 
    $b+n=a$. Then we have $b+n+c=b+c+n=(b+c)+n=a+c$. Thus, $a+c\geq b+c$. Then, we need to show that $a+c\geq b+c\Rightarrow a\geq b$. Since $a+c\geq b+c$, there should be a natural number $n$ such 
    that $b+c+n=b+n+c=(b+n)+c=a+c$. By Cancellation law, we have $b+n=a$. Thus, $a\geq b$. 

    Thus, if $a\geq b$ and $b\geq a$, then $a=b$.
\end{proof}
\subsubsection*{(e)}
$a<b$ if and only if $a++\leq b$.
\begin{proof}
    First, we need to show that $a<b\Rightarrow a++\leq b$. $a<b$ means there exists a natural number $n$ such that 
    $a+n=b$, particularly, $a\ne b$. Then $n$ must not be zero. So $n$ is the predecessor of a natural number, denote it as $m$. Then we have $a+n=a+(m++)=(a+m)++=(a++)+m=b$. Therefore, $a++\leq b$. Then 
    we need to show that $a++\leq b\Rightarrow a<b$. There exists a natural number $n$ such that $(a++)+n=b$. $(a++)+n=(a+n)++=a+(n++)=b$. Since $n++$ is the successor of $n$, $n++$ must not be equal to $0$.
    If $a=b$, there will be $a+(n++)=a\Rightarrow n++=0$, contradiction. Therefore, $a\ne b$. 
    
    Thus, $a<b$ if and only if $a++\leq b$.
\end{proof}
\subsubsection*{(f)}
$a<b$ if and only if $b=a+d$ for some positive number $d$.
\begin{proof}
    First, we need to show that $a<b\Rightarrow b=a+d$ for some positive number $d$. There exists some natural number $d$ such that
    $a+d=b$, $a\ne b$. By Cancellation law, $d$ must not be zero. Therefore, $d$ is positive. Then, we need to show that $a+d=b$ for some positive $d\Rightarrow a<b$. We only need to prove $a\ne b$.
    If $a=b$, we have $a+d=a=a+0$. By Cancellation law, $d=0$ which contradicts to $d$ is positive. Therefore, $a\ne b$.
    
    Thus, $a<b$ if and only if $b=a+d$ for some positive number $d$.
\end{proof}
\subsection*{2.2.4}
Justify the three statements marked in the proof of Proposition 2.2.13. 
\subsubsection*{(a)}
$0\leq b$ for all $b$.
\begin{proof}
    By definition of addition, we have $0+b=b$. Thus, $0\leq b$.
\end{proof}
\subsubsection*{(b)}
If $a>b$, then $a++>b$.
\begin{proof}
    By Proposition 2.2.12.e, we have $a>b\Rightarrow a\geq b++$. And by Proposition 2.2.12.d, $a+1\geq (b++)+1$ that is equivalent to 
    $a++\geq b+2$. Since 2 is positive, by Proposition 2.2.12.f, we have $a++>b$. 
\end{proof}
\subsubsection*{(c)}
If $a=b$, then $a++>b$.
\begin{proof}
    We know from Proposition 2.2.12.a that $a\geq a$, so $a\geq a=b$. And again by Proposition 2.2.12.d, we have $a++=a+1\geq b+1$. Since 1 is positive, 
    by Proposition 2.2.12.f, $a++>b$.

\end{proof}
\subsection*{2.2.5}
Proposition 2.2.14 (Strong principle of induction). Let $m_0$ be a natural number, and let $P(m)$ be a property pertaining to an arbitrary natural 
number $m$. Suppose that for each $m\geq m_0$, we have the following implication: if $P(m')$ is true for all natural numbers $m_0\leq m'<m$, then $P(m)$ is also true.
Then we can conclude that $P(m)$ is true for all natural numbers $m\geq m_0$.
\begin{proof}
    Let $Q(n)$ be the property that $P(m)$ is true for all $m_0\leq m<n$. Induct on $n$. 
    Consider the base case $n=0$. This is vacuously true. In fact, $Q(n)$ is vacuously true for all $n \leq m_0$. So we can assume $n>m_0$ to see if the implication stands. Suppose $Q(n)$ is true, that is, $P(m)$ is true for all $m_0\leq m<n$. 
    We want to show that $Q(n+1)$ is also true. As stated in Proposition 2.2.14, if $Q(n)$ is true, then $P(n)$ is also true. 
    So $P(m)$ is true for all $m_0\leq m\leq n$. Hence, $P(m)$ is true for all $m_0\leq m<n+1$. ($m\leq n \Leftrightarrow m<n+1$ can be shown using prop 2.2.12.) Thus, $Q(n+1)$ is true. This closes the induction.
\end{proof}
\subsection*{2.2.6}
Let $n$ be a natural number, and let $P(m)$ be a property pertaining to the natural numbers such that whenever $P(m++)$ is true, then $P(m)$
is true. Suppose that $P(n)$ is also true. Prove that $P(m)$ is true for all natural numbers $m\leq n$. (Principle of backwards induction.)
\begin{proof}
    Apply induction to $n$. For the base case $n=0$, suppose $P(0)$ is true. In this case, $m$ can only be 0 ($m+k=0 \Rightarrow m=0,k=0$). Since $P(0)$ is true, 
    the base case is proved. Next, suppose if $P(n)$ is true then $P(m)$ is true for all natural numbers $m\leq n$. We want to show that if $P(n++)$ is true then $P(m)$ is true for all 
    natural numbers $m\leq n++$. $m\leq n$ means there exists a natural number $a$ such that $m+a=n++$. $a$ is either 0 or a positive number. If $a$ is 0, $m=n++$. If $a$ is positive, $m<n++$ (by prop 2.2.12.f), this is equivalent to $m\leq n$ (can be shown using prop 2.2.12.). 
    For $m=n++$, $P(m)$ is true because of the assumption. For each $m\leq n$, $P(m)$ is also true by induction hypothesis. Therefore, $P(m)$ is true for all natural numbers $m\leq n++$. And we have closed the induction.
\end{proof}
In the above proofs, $n++$ and $n+1$ got mixed up because $n++=n+1$ has been illustrated on Page 26 (and the $+1$ version is a little easier). But $++$ is a more desirable expression since it stands for the successor in a general way.

\section*{2.3}
\subsection*{Definition 2.3.1 (Multiplication of natural numbers).}
Let $m$ be a natural number. To multiply zero to $m$, we define $0\times m:=0$. Now suppose inductively that we have defined how to multiply $n$ to $m$. Then we can multiply $n++$ to $m$ by defining $(n++)\times m:=(n\times m)+m$.
\subsection*{2.3.1}
\textbf{Lemma 2.3.2} (Multiplication is commutatitive). Let $n,m$ be natural numbers. Then $n\times m=m\times n$.
\begin{proof}
    First, we want to show that $m\times 0=0$. Induct on $m$. When $m=0$, by definition $0\times m=0$ for every $m$, so $0\times 0=0$. Suppose $m\times 0=0$, we want to show that $(m++)\times 0=0$. By definition, we got $(m++)\times 0=(m\times 0)+0$ which is equal to $0+0=0$. 
    This closes the induction.

    Then, we want to show that $n\times(m++)=n\times m+n$. Induct on $n$ by keeping $m$ fixed. Consider the base case $n=0$. The LHS is equal to $0\times(m++)=0$ by definition. The RHS is equal to 
    $0\times m+0$ which is also 0. Now suppose inductively $n\times (m++)=n\times m+n$. We need to show that $(n++)\times (m++)=(n++)\times m+(n++)$. 
    \begin{equation*}
        \begin{aligned}
            \text{LHS}&=(n++)\times(m++)=n\times (m++)+(m++)\\&=n\times m+n+(m++)=n\times m+(n+m)++,\\
            \text{RHS}&=(n++)\times m+(n++)=n\times m+m+(n++)=n\times m+(n+m)++.
        \end{aligned}
    \end{equation*}
    Thus, both sides are equal to each other. This closes the induction.

    Now we can use the things above to show Lemma 2.3.2. We induct on $n$ by keeping $m$ fixed. Consider the base case $n=0$. $0\times m=m\times 0=0$ by definition and the lemma we have shown above. Assume inductively $n\times m = m\times n$. We want to show that 
    $(n++)\times m=m\times (n++)$. By definition, the LHS is equal to $(n++)\times m=(n\times m)+m$. By the lemma we proved above, the RHS is equal to $m\times(n++)=m\times n+m=n\times m+m$. So both sides are equal to each other. We have closed the induction.  
\end{proof}
\subsection*{2.3.2}
\textbf{Lemma 2.3.3} (Positive natural numbers have no zero divisors). Let $n,m$ be natural numbers. Then $n\times m=0$ if and only if at least one of $m,n$ is equal to zero. In particular, if $n$ and $m$ are both positive, then $nm$ is also positive.
\begin{proof}
    Try to prove the second statement first. Assume $n,m$ are both positive natural numbers. So we can represent $n$ as $a++$ where $a$ is a natural number. Then 
    \begin{equation*}
        \begin{aligned}
            nm&=(a++)\times m\\
            &=a\times m+m.
        \end{aligned}
    \end{equation*}
    Since $m$ is positive and $a\times m$ is at least 0, $nm=a\times m+m$ must be positive. In this sense, we have shown $n\times m=0\Rightarrow$ at least one of $n,m$ is zero since\\ 
    $p\rightarrow q\equiv ~q\rightarrow ~p$. The rest part is to show at least one of $n,m$ is zero $\Rightarrow n\times m=0$. This is trivial and can be directly proved using the definition. 

    Thus, $n\times m=0$ if and only if at least one of $m,n$ is equal to zero.
\end{proof}
\subsection*{2.3.3}
\textbf{Proposition 2.3.5} (Multiplication is associative). For any natural numbers $a,b,c$, we have $(a\times b)\times c=a\times (b\times c)$.
\begin{proof}
    Fix $a,c$ and induct on $b$. Consider the base case when $b=0$. 
    \begin{equation*}
        \begin{aligned}
            \text{LHS}=(a\times 0)\times c=0\times c=0,\\
            \text{RHS}=a\times(0\times c)=a\times 0=0.
        \end{aligned}
    \end{equation*}
    Thus, the base case is proved. Assume inductively $(a\times b)\times c=a\times(b\times c)$. We need to show that $(a\times(b++))\times c=a\times((b++)\times c)$.
    \begin{equation*}
        \begin{aligned}
            \text{LHS}&=(a\times(b++))\times c=(a\times b+a)\times c=(a\times b)\times c+ac,\\
            \text{RHS}&=a\times ((b++)\times c)=a\times(b\times c+c)=a\times (b\times c)+ac.
        \end{aligned}
    \end{equation*}
    By induction hypothesis, $(a\times b)\times c=a\times(b\times c)$. Thus, both sides are equal to each other. This closes the induction.
\end{proof}
\subsection*{2.3.4}
Prove the identity $(a+b)^2=a^2+2ab+b^2$ for all natural numbers $a,b$.
\begin{proof}
    Suppose $a$ is an arbitrary natural number and keep $a$ fixed. Induct on $b$. First consider the base case $b=0$. 
    \begin{equation*}
        \begin{aligned}
            \text{LHS}&=(a+0)^2=a^2,\\
            \text{RHS}&=a^2+2ab+b^2=a^2+0+0=a^2.
        \end{aligned}
    \end{equation*}
    So the base case is proved. Now assume inductively $(a+b)^2=a^2+2ab+b^2$. We need to show that $(a+(b++))^2=a^2+2a(b++)+(b++)^2$.
    \begin{equation*}
        \begin{aligned}
            \text{LHS}&=(a+(b++))^2\\&=((a+b)++)^2\\
            &=((a+b)++)\times((a+b)++)\\
            &=(a+b)(a+b)+(a+b)+(a+b)++\\
            &=\underbrace{a^2+2ab+b^2}_{\text{by induction hypothesis}}+(2a+2b)++,\\
            \text{RHS}&=a^2+2a(b++)+(b++)^2\\
            &=a^2+2ab+2a+b(b++)+(b++)\\
            &=a^2+2ab+2a+b^2+b+(b++)\\
            &=a^2+2ab+b^2+(2a+2b)++.
        \end{aligned}
    \end{equation*}
    Thus, both sides are equal to each other. This closes the induction.
\end{proof}
\subsection*{2.3.5}
\textbf{Proposition 2.3.9} (Euclidean algorithm). Let $n$ be a natural number, and let $q$ be a positive number. Then there exist natural numbers $m,r$ such that 
$0\leq r<q$ and $n=mq+r$.
\begin{proof}
    Fix $q$ and induct on $n$. Consider the base case $n=0$. Let $m=0,r=0$, then $mq+r=0\times q+0=0$ as required. Now assume inductively there exist natural numbers $m,r$ such that 
    $0\leq r<q$ and $n=mq+r$. What we want to show is there exist natural numbers $m',r'$ such that $0\leq r'<q$ and $n+1=m'q+r'$. Since $r<q$, we have two cases: $r+1<q$ and $r+1=q$. \\
    Case 1: $r+1<q$. Let $m'=m$, $r'=r+1$, $0\leq r'<q$. Then $m'q+r'=mq+(r+1)=(mq+r)+1=n+1$ as required. \\
    Case 2: $r+1=q$. $n+1=mq+(r+1)$ since $n=mq+r$ by induction hypothesis. Substitute $(r+1)$ with $q$, we have $n+1=mq+q=(m+1)q$. Let $m'=m+1,r'=0$. We got $n+1=m'q+r'$ as required.  
    We got $n+1=m'q+r'$ as required. Thus, we have closed the induction.
\end{proof}
\end{document}